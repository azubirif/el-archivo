\documentclass{article}
\author{Alejandro Zubiri}
\title{hooke}

\renewcommand*\contentsname{Índice}

\usepackage[margin=1.1in]{geometry}
\usepackage{amsmath, physics, amsthm, amsfonts}

\newtheorem{teorema}{Teorema}
\newtheorem{defin}{Definición}

\newcommand{\R}{\mathbb{R}}

\begin{document}
\maketitle
\tableofcontents
\pagebreak
\section{Introducción}
Se va a estudiar como pruebas experimentales acerca de la ley de Hooke concuerdan con la teoría. Para ello, se han hecho diferentes experimentos, donde se estudiaban diferentes movimientos, muelles, y pesas.
\section{Teoría}
La ley de Hooke afirma que la fuerza ejercida por un muelle de constante elástica $K$, sobre un objeto alejado una distancia $\Delta \vec{r}$ de la posición de equilibrio del muelle viene dada por la ecuación:
\begin{equation}
	\begin{split}
		\boxed{\vec{F} = -k\Delta \vec{r}}
	\end{split}
\end{equation}
Para una dimensión, y sobre un objeto cuya única forma es el muelle, se obteniene la siguiente EDO:
\begin{equation}
	\begin{split}
		m\dv[2]{x}{t}=-kx
	\end{split}
\end{equation}
Que tiene como solución:
\begin{equation}
	\begin{split}
		x(t)=A\sin (\sqrt{\frac{k}{m}}t)
	\end{split}
\end{equation}
\section{Material y procedimiento}
\begin{itemize}
\item Dos muelles de diferentes constantes elásticas.
\item Dos punteros de plástico para medir la posición.
\item Una regla milimetrada.
\item Una barra soporte que sujetará los muelles.
\item Base reguladora donde se apoyaría la barra soporte.
\item Horquilla de donde colgar los muelles.
\item Balanza de donde medir las pesas antes de colgarlas.
\item Cronómetro para medir los períodos de los movimientos.
\item Una serie de pesas y porta-pesas.
\end{itemize}
\section{Procedimiento}
Para confirmar la ley de Robert Hooke, se han realizado dos tipos de experimentos. El primero ha consistido en colgar un muelle, y desde la punta, colgar una pesa de masa $m$. Se estudiará la elongación $\Delta y$ del muelle respecto a su posición de equilibrio. Para la segunda parte del experimento, desde la posición de equilibrio, se estiró (hacia arriba o hacia abajo) una distancia $A$. Luego, se observó el tiempo que tardaba la pesa en realizar 10 oscilaciones (para mayor precisión) y se dividió entre $ 10 $ para obtener el período de una oscilación individual.
\subsection{Pesa colgada de muelle}
Este primer experimento otorgó los siguientes resultados:
\begin{table}[h]
	\centering

	\begin{tabular}{c | c | c | c | c | c}
		
	\end{tabular}
\end{table}
\end{document}_
