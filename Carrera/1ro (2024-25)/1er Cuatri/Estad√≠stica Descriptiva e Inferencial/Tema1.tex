\documentclass{article}
\author{Alejandro Zubiri}
\date{Wed Oct 09 2024}
\title{Tema 1}

\usepackage{amsmath}

\begin{document}
\maketitle
\tableofcontents
\pagebreak
\section{Definiciones Básicas}
\begin{itemize}
    \item \textbf{Muestra}: datos recopilados
    \item \textbf{Escalas}: intervalo (existe 0 pero no indica ausencia) y razón (0 es ausencia).
\end{itemize}

\section{Tabla de distribución}
Se trabaja con la \textbf{frecuencia} con la que aparecen los datos:
\begin{itemize}
    \item Absoluta $(f)$: cuánto aparece.
    \item Total $(n)$: número de datos.
    \item Relativa $(f_r)$: $\frac{f}{n}$.
    \item Acumulada: con datos ordenados, sumar todos los anteriores a $x_i$:
    \item Absoluta $(F)$
    \item Relativa $(F_r)$
\end{itemize}
\section{Clases}
\begin{itemize}
    \item Clase modal: clase que tiene más frecuencia por unidad de amplitud (Densidad de frecuencia).\\
    se calcula mediante
    \begin{equation}
        \begin{split}
            h_j= \frac{f_j}{b_j - b_{j-1}}
        \end{split}
    \end{equation}
    Es decir, la frecuencia de la clase entre su amplitud.
    \item Intervalo / Clase mediana: clase donde se encuentra la mediana.
\end{itemize}
\section{Representación gráfica}
\subsection{Función de distribución empírica}
La función de distribución empírica se define por intervalos. En una gráfica, haremos tantos
puntos en el eje X como extremos de intervalos haya. Al final del intervalo, su altura será
su frecuencia relativa acumulada. Después, uniremos todos los puntos.
\section{Medidas de centralización}
\subsection{Media aritmética}
\begin{equation}
    \begin{split}
        \bar{x}=\frac{\sum x_i}{n}
    \end{split}
\end{equation}
Para datos agrupados en clases, siendo $m_j$ el valor central y $f_r(m_j)$ la frecuencia relativa de esta:
\begin{equation}
    \begin{split}
        \bar{x}=\sum m_j f_r(m_j)
    \end{split}
\end{equation}
Para \textbf{transformaciones a la media}, esta es linear, puesto que para una suma de $k$, se le suma $k$ a la media,
y una transformación por $k$ multiplica po $k$ la media.
\subsection{Media geométrica}
\begin{equation}
    \begin{split}
        \bar{x}_G = (\prod x_i)^{\frac{1}{n}}
    \end{split}
\end{equation}
\subsection{Media armónica}
\begin{equation}
    \begin{split}
        \bar{x}_H=\frac{n}{\sum \frac{1}{x_{i}}}
    \end{split}
\end{equation}
Es fácilmente demostrable que:
\begin{equation}
    \begin{split}
        \bar{x}_H\leq\bar{x}_G\leq\bar{x}
    \end{split}
\end{equation}
\subsection{Mediana}
La mediana es el valor que está ``en medio`` en un número de datos impar, o la media aritmética de los dos datos del centro.
Cumple que el $50\%$ de los valores es menor a la mediana y el otro $50\%$ es mayor.
\subsection{Moda}
Es el valor más frecuente.
\section{Medidas de dispersión}
Miden la separación de los datos entre sí
\subsection{Desviación media}
\begin{equation}
    \begin{split}
        D_{\bar{x}}=\frac{1}{n}\sum|x_{i}-\bar{x}|
    \end{split}
\end{equation}
\subsection{Varianza}
\begin{equation}
    \begin{split}
        s^{2}_{x}=\frac{\sum(x_{i}-\bar{x})^{2}f(x_{i})}{n}=\frac{1}{n}\sum x_{i}^{2}f(x_{i})-\bar{x}^{2}
    \end{split}
\end{equation}
Algunas propiedades de la varianza:
\begin{itemize}
    \item Es acotado y positivo
    \item No se ve afectado por cambios de origen
    \item Se ve afectado por cambios de escala $k$ en un factor $k^{2}$. 
\end{itemize}
\subsection{Cuasivarianza}
\begin{equation}
    \begin{split}
        \hat{s}^{2}_{x}=\frac{\sum(x_{i}-\bar{x})^{2}f(x_{i})}{n-1}
    \end{split}
\end{equation}
\subsection{Desviación típica}
\begin{equation}
    \begin{split}
        \bar{s}_{x}=\frac{\sqrt{ \sum(x_{i}-\bar{x})^{2}f(x_{i}) }}{n}
    \end{split}
\end{equation}
\subsection{Cuasidesviación típica}
\begin{equation}
    \begin{split}
        \hat{s}_{x}=\frac{\sqrt{ \sum(x_{i}-\bar{x})^{2}f(x_{i}) }}{n-1}
    \end{split}
\end{equation}
\subsection{Coeficiente de variación}
Mide la dispersión relativamente a los datos.
\begin{equation}
    \begin{split}
        CV=\frac{s_{x}}{|\bar{x}|} / \bar{x}\neq 0
    \end{split}
\end{equation}
\subsection{MEDA}
"Mediana de las desviaciones absolutas respecto a la media"
\begin{equation}
    \begin{split}
        MEDA=median(|x_{i}-Med_{x})
    \end{split}
\end{equation}
\subsection{Cuantiles}
Dividen la distribución en $c$ partes. Los más comunes son:
\begin{itemize}
    \item Cuartiles $(Q)$: cuatro partes
    \item Quintiles $(K)$: cinco partes
    \item Percentiles $(p)$: cien partes
\end{itemize}
La posición de un cuantil viene dada por:
\begin{equation}
    \begin{split}
        C_i= \frac{i \cdot n}{c}
    \end{split}
\end{equation}
El cuantil $i$, con $n$ datos y $c$ divisiones.\\
Para encontrar el valor del cuantil, que definiremos como $C_i^v$, tenemos que encontrar el valor
cuya frecuencia absoluta acumulada sea mayor o igual al valor de la posición cuantil:
\begin{equation}
    \begin{split}
        C_i^v \to F_i \geq C_i
    \end{split}
\end{equation}
\subsection{Rango Intercuartílico}
\begin{equation}
    \begin{split}
        RI=Q_3-Q_1
    \end{split}
\end{equation}
\section{Medidas de forma}
\subsection{Coeficiente de asimetría Fisher}
Caracteriza la deformación en el eje X:
\begin{equation}
    \begin{split}
        CA_f=\gamma_1= \frac{\sum {(x_i-\bar{x})}^3 f(x_{i})}{ns^3}
    \end{split}
\end{equation}
\begin{itemize}
    \item Si $\gamma_1=0\Rightarrow$ distribución simétrica. La media es igual a la mediana. 
    \item Si $\gamma_1>0\Rightarrow$ distribución asimétrica a derechas. La media es mayor a la mediana.
    \item Si $\gamma_1<0\Rightarrow$ distribución asimétrica a izquierdas. La media es menor a la mediana.  
\end{itemize}
\section{Medidas de concentración}
\subsection{Curva de Lorenz}
Definimos los montos acumulados $S_i$ como
\begin{equation}
    \begin{split}
        S_i=x_i f_i
    \end{split}
\end{equation}
Y $S_n$ como
\begin{equation}
    \begin{split}
        S_n=\sum x_i f_i
    \end{split}
\end{equation}
Definimos $q_i$
\begin{equation}
    \begin{split}
        q_i = \frac{S_i}{S_n} \cdot 100
    \end{split}
\end{equation}
Y $p_i$
\begin{equation}
    \begin{split}
        p_i= \frac{F_i}{n} \cdot 100
    \end{split}
\end{equation}
Ahora, las coordenadas de la curva serán
\begin{equation}
    \begin{split}
        \frac{p_i}{q_i}
    \end{split}
\end{equation}
\subsection{Índice de Gini}
Oscila entre $0$ y $1$, el 0 indicando uniformidad, y el 1 máxima desigualdad. Es la medida más
apropiada para la concentración.
\begin{equation}
    \begin{split}
        I_G= \frac{\sum^{n-1} (p_i-q_i)}{\sum ^{n-1} p_i}= 1-\frac{\sum^{n-1} q_i}{\sum ^{n-1} p_i}
    \end{split}
\end{equation}
\subsection{Coeficiente de Asimetría Bowley}
\begin{equation}
    \begin{split}
        CA_B= \frac{Q_1+Q_3-2Q_2}{RI}
    \end{split}
\end{equation}
\subsection{Curtosis}
Agrupamiento respecto a la media
\begin{equation}
    \begin{split}
        CA_P= \frac{\sum (x_i-\bar{x})^4}{ns^4}-3
    \end{split}
\end{equation}
\subsection{Diagrama de Caja}
Necesitamos los mínimos, máximos y cuartiles.\\
Definimos el límite inferior como
\begin{equation}
    \begin{split}
        L_I= Q_1^v-1.5RI
    \end{split}
\end{equation}
Y el límite superior como
\begin{equation}
    \begin{split}
        L_S=Q_3^v+1.5RI
    \end{split}
\end{equation}
Todos los datos por debajo del inferior o por encima del superior son atípicos.
\subsection{Números Índice}
Cambio relativo de una variable o variables respecto al tiempo. Se relaciona el valor actual con el valor en un período base.
\begin{equation}
    \begin{split}
        I= \frac{x_t}{x_0}
    \end{split}
\end{equation}
Donde $x_t$ es el valor ahora y $x_0$ el valor en el período base.
\subsection{Índice de Laspeyres}
Sea
\begin{itemize}
    \item $q_{i0}$ la cantidad comprada en el período origen.
    \item $p_{i0}$ el precio de producto $i$ en tiempo origen.
    \item $q_{it}$ la cantidad de $i$ comprada ahora.
    \item $p_{it}$ el precio de $i$ ahora.
\end{itemize}
\begin{equation}
    \begin{split}
        IPL_t= \frac{\sum p_{it}q_{i0}}{\sum p_{i0}q_{i0}}
    \end{split}
\end{equation}
\subsection{Índice de Paasche}
Con la mismas variables
\begin{equation}
    \begin{split}
        IPP_t = \frac{\sum p_{it} q_{it}}{\sum p_{i0} q_{it}}
    \end{split}
\end{equation}
\end{document}
