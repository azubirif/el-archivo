\documentclass{article}
\author{Alejandro Zubiri}
\date{Thu Oct 17 2024}
\title{Topología}

\usepackage{amsmath, amsthm, amsfonts, multicol}

\begin{document}
\maketitle
\begin{multicols}{2}
    \section{Definiciones}
    \subsection{Distancia}
    Dado un conjunto $E$ y una métrica, una distancia es una aplicación:
    \begin{equation}
        \begin{split}
            d: E \times E \to \mathbb{R}^+ \cup \{ 0 \}
        \end{split}
    \end{equation}
    \subsection{Espacio métrico}
    Un espacio métrico es un par $(E,d)$, donde $E$ es un conjunto y $d$ una función distancia.
    \subsection{Bola abierta}
    Una bola abierta es el conjunto de puntos que se encuentra a una distancia $<r$ de un centro $x_{0}$:
    \begin{equation}
        \begin{split}
            B(x_{0},r)= \{ x / d(x,x_{0})<r \}
        \end{split}
    \end{equation}
    \subsection{Bola cerrada}
    \begin{equation}
        \begin{split}
            \bar{B}( x_{0},r)= \{ x / d(x,x_{0}) \leq r \}
        \end{split}
    \end{equation}
    \subsection{Entorno}
    Un subconjunto $A \subset E$ es un entorno si existe una bola abierta contenida en $A$.
    \subsection{Bola abierta y entorno reducidos}
    Dada una bola $B(x_{0},r)$ y un entorno $A$, la bola reducida es $B(x_{0},r)-\{ x_{0} \}$ y
    $A-\{ x_{0} \}$.
    \subsection{Punto interior}
    Un punto $x_{0} \in E$ es interior a $A$ si $\exists r > 0 / B(x_{0},r) \subset A$
    \subsection{Punto exterior}
    Un punto $x_{0} \in E$ es exterior a $A$ si $\exists r > 0 / B(x_{0},r) \subset \bar{A}$
    \subsection{Punto de acumulación}
    $x_{0} \in E$ es de acumulación de $A$ si
    \begin{equation}
        \begin{split}
            \forall r > 0 [B(x_{0},r)-\{ x_{0} \}] \cap A \neq \O
        \end{split}
    \end{equation}
\end{multicols}{2}
\end{document} 