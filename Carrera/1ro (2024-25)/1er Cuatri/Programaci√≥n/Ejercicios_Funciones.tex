\documentclass{article}
\author{Alejandro Zubiri}
\date{Mon Nov 11 2024}
\title{Ejercicios Funciones}

\begin{document}
\maketitle
\section*{Requisitos para las funciones}
\begin{itemize}
    \item El nombre de la función debe ser \textbf{exactamente igual} al mostrado en el enunciado,
    y debe tener exactamente los mismos parámetros y en el mismo orden.
    \item Si el enunciado pide un programa que imprima o devuelva un valor debe ser así, no
    se puede cambiar.
\end{itemize}
\pagebreak
\section{Función que saluda}
\begin{itemize}
    \item \textbf{Función}: \verb|saluda(nombre)|
    \item \textbf{Parámetros}: \verb|nombre|
\end{itemize}
Crea una función que dado un parámetro (un nombre), imprima la siguiente cadena de texto:
\begin{verbatim}
    >>> ¡Hola {nombre}!
\end{verbatim}
\begin{verbatim}
    >>> saluda("Carlos")
    ¡Hola Carlos!
    >>> saluda("Yerali")
    ¡Hola Yerali!
\end{verbatim}
\section{Función que imprima una pirámide}
\begin{itemize}
    \item \textbf{Función}: \verb|piramide(anchura)|
    \item \textbf{Parámetros}: \verb|anchura|
\end{itemize}
Esta función debe imprimir una pirámide de \# cuya anchura máxima sea
el parámetro \verb|anchura|:
\begin{verbatim}
    >>> piramide(4)
    #
    ##
    ###
    ####
\end{verbatim}
\pagebreak
\section{Número repetido}
\begin{itemize}
    \item \textbf{Función}: \verb|numero_repetido(cadena, numero)| 
    \item \textbf{Parámetros}: \verb|cadena|: serie de numeros \textbf{de tipo entero}.
    \verb|numero|: el número que queremos saber si se repite.
    \item \textbf{Devuelve}: \verb|True| si se repite el número, \verb|False| si no se repite.
\end{itemize}
Crea una función que cuando se le pase un número entero como primer argumento, devuelva \verb|True|
o \verb|False| si el número está \textbf{2 o más veces} en la cadena original:
\begin{verbatim}
    >>> numero_repetido(12345, 5)
    False
    >>> numero_repetido(1223, 2)
    True
    >>> numero_repetido(1335, 5)
    False
\end{verbatim}
\section{Vocales al revés}
\begin{itemize}
    \item \textbf{Función}: \verb|vocales_al_reves(cadena)|
    \item \textbf{Parámetros}: \verb|candena|: la cadena a analizar
    \item \textbf{Devuelve}: las vocales que hay en la cadena, pero al revés.
\end{itemize}
Crea una función que cuando se le pase una cadena de texto, devuelva las vocales de la cadena, pero al revés. Si
la cadena no tiene vocales, que devuelva una cadena vacía.
\begin{verbatim}
    >>> vocales_al_reves("hola")
    "ao"
    >>> vocales_al_reves("pedro")
    "oe"
    >>> vocales_al_reves("xyz")
    ""
\end{verbatim}
\pagebreak
\section{Cadena palíndroma}
\begin{itemize}
    \item \textbf{Función}: \verb|es_palindroma(cadena)|
    \item \textbf{Parámetros}: \verb|cadena|: la cadena a analizar
    \item \textbf{Devuelve}: \verb|True| si la cadena es palíndroma, y \verb|False| si no lo es.
\end{itemize}
Una cadena palíndroma es una cadena que se lee igual que si está al revés. Ejemplos son "ojo", "somos", "seres".
La función también debería funcionar para números.
\begin{verbatim}
    >>> es_palindroma("ojo")
    True
    >>> es_palindroma("pan")
    False
    >>> es_palindroma(1221)
    True
\end{verbatim}
\section{Hipotenusa}
\begin{itemize}
    \item \textbf{Función}: \verb|hipotenusa(cateto_a, cateto_b)|
    \item \textbf{Parámetros}: los lados del triángulo.
    \item \textbf{Devuelve}: la longitud de la hipotenusa (no su cuadrado)
\end{itemize}
Esta función debe tomar los lados de un triángulo \textbf{rectángulo} y calcular su hipotenusa:
\begin{verbatim}
    >>> hipotenusa(3, 4)
    5.0
    >>> hipotenusa(3.14, 2.78)
    4.193804954930546
\end{verbatim}
\end{document}