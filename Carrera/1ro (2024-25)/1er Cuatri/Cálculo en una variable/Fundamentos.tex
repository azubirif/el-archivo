\documentclass{article}
\author{Alejandro Zubiri}
\date{Thu Nov 21 2024}
\title{Fundamentos}

\usepackage{amsmath, amsthm, amsfonts}

\begin{document}
\maketitle
\tableofcontents
\pagebreak

\section{Conjuntos}
\subsection{Cota superior}
Sea $A$ un subconjunto de $\mathbb{R}$. Se dice que $A$ está acotado superiormente si existe un
subconjunto $M \subset \mathbb{R}:$
\begin{equation}
    \begin{split}
        \forall k \in M, x \in A, k \geq x
    \end{split}
\end{equation}
Es decir, que cualquier elemento perteneciente a $M$ es mayor que cualquier elemento de $A$.
\subsection{Cota inferior}
Similarmente, se dice que un subconjunto $A$ está acotado inferiormente si:
\begin{equation}
    \begin{split}
        \forall k \in M, x \in A, x \geq k
    \end{split}
\end{equation}
\begin{proof}[Supremo]
    El supremo de un subconjunto $A$ es el valor más pequeño del conjunto de cotas superiores.
\end{proof}
\begin{proof}[Ínfimo]
    El ínfimo de un subconjunto $A$ es el valor más grande del conjunto de cotas inferiores.
\end{proof}
\begin{proof}[Máximo]
    Es el valor más pequeño del conjunto que \textbf{también} pertenece al conjunto de
    cotas superiores.
\end{proof}
\begin{proof}[Mínimo]
    Es el valor más grande del conjunto que, además, pertenece al conjunto de cotas inferiores.
\end{proof}
\end{document}