\documentclass{article}
\author{Alejandro Zubiri}
\date{Fri Oct 11 2024}
\title{Funciones}

\usepackage{amsmath}
\usepackage{amsthm}
\usepackage{physics}
\usepackage{amsfonts}

\newtheorem{deriv}{Definición}
\newtheorem{derivableLuegoContinua}{Teorema}
\newtheorem{rolle}{Teorema de Rolle}[section]
\newtheorem{midval}{Teorema del punto medio}
\newtheorem{derivada_par_impar}{Teorema}
\newtheorem{diferencial}{Definición}

\begin{document}
\maketitle
Una función es una correspondencia que a cada valor de un conjunto, le asigna otro único valor de otro conjunto. En este
caso, se estudiarán funciones \textbf{reales}:
\begin{equation}
    \begin{split}
        f(x): \mathbb{R} \to \mathbb{R}
    \end{split}
\end{equation}
Para representar una función, utilizamos un \textbf{grafo}, que es una representación de todos valores que $f$ toma
dependiendo del valor que toma $x$.\\
El conjunto \textbf{dominio} es el conjunto de valores para los cuales $\exists f(x)$.\\
El conjunto \textbf{imagen} es el conjunto de todos los valores que puede tomar $f(x)$.
\begin{equation}
    \begin{split}
        &\mathrm{Dom}(f(x))= \{ x / \forall x \in \mathbb{R}, \exists f(x) \}\\
        &\mathrm{Im}(f(x))= \{ y / \forall x,y \in \mathbb{R}, y=f(x) \} 
    \end{split}
\end{equation}
Distinguimos entre funciones:
\begin{itemize}
    \item Pares: $f(x) = f(-x)$
    \item Impares: $f(x)=-f(-x)$
    \item Periódicas: $\sin x, \cos x\dots$ 
\end{itemize}
\pagebreak
\section{Continuidad y derivabilidad}
\begin{deriv}[Derivada]
    La derivada de una función $f(x)$ en un punto $x=a$ es la tasa de cambio
    instantánea de $f(x)$ con respecto a $x$:
    \begin{equation}
        \begin{split}
            f'(x)= \lim_{h \to 0} \frac{f(x+h)-f(x)}{h}
        \end{split}
    \end{equation}
\end{deriv}
\begin{deriv}[Diferencial]
    El diferencial $ \dd{f}$ de una función $f$ es un incremento infinitesimal a lo largo de esta:
    \begin{equation}
        \begin{split}
            \dd{f} = \lim_{x \to x_{0}}f(x)-x_{0}&=\lim_{x \to x_{0}}f(x)-x_{0}\cdot \frac{x-x_{0}}{x-x_{0}}
            =f'(x_{0}) \dd{x}\\
            \dd{f} &= f'(x_{0}) \dd{x}
        \end{split}
    \end{equation}
\end{deriv}
Una función $f(x)$ es continua en un punto $x=a$ si se cumple que:
\begin{itemize}
    \item $\exists f(a)$
    \item $\exists \lim_{x \to a} f(x) \Rightarrow \lim_{x \to a^-} f(x)= \lim_{x \to a^+} f(x)$
    \item $\lim_{x \to a}f(x)=f(a)$  
\end{itemize}
Similarmente, una función es derivable en un punto $x=a$ si se cumple que:
\begin{equation}
    \begin{split}
        \exists \lim_{x \to a} f'(x) \Rightarrow \lim_{x \to a^-} f'(x)= \lim_{x \to a^+} f'(x)
    \end{split}
\end{equation}
\begin{derivableLuegoContinua}[Teorema]
    Si una función es derivable en un punto $x=a$, entonces también es continua.
\end{derivableLuegoContinua}
\begin{proof}[Demostración]
    Queremos demostrar que
    \begin{equation}
        \begin{split}
            \lim_{x \to a} f(x)=f(a) \iff \lim_{x \to a} f(x)-f(a)=0
        \end{split}
    \end{equation}
    que es la definición de continuidad.\\
    Partimos de que nuestra función es derivable, lo que implica que
    \begin{equation}
        \begin{split}
            \exists f'(a) \Rightarrow \lim_{x \to} \frac{f(x)-f(a)}{x-a} \in \mathbb{R}
        \end{split}
    \end{equation}  
    \begin{equation}
        \begin{split}
            \lim_{x \to a} f(x)-f(a) = \lim_{x \to a} (f(x)-f(a)) \cdot \frac{x-a}{x-a}
            = \lim_{x \to a} \frac{f(x)-f(a)}{x-a} \cdot \lim_{x \to a} x-a
        \end{split}
    \end{equation}
Podemos identificar que el primer término es la definición de derivada, que sabemos que existe y
que es menor a $\infty$. El segundo término tiende a $0$.
\begin{equation}
    \begin{split}
        \lim_{x \to a} \frac{f(x)-f(a)}{x-a} \cdot \lim_{x \to a} x-a =f'(a) \cdot 0=0
    \end{split}
\end{equation}
Obteniendo así que
\begin{equation}
    \begin{split}
        \lim_{x \to a} f(x) -f(a)=0
    \end{split}
\end{equation}
QED
\end{proof}
\section{Representación de funciones}
Para representar una función, debemos encargarnos de una serie de apartados:
\subsection{Dominio}
Debemos calcular el conjunto dominio de la función:
\begin{equation}
    \begin{split}
        \mathrm{Dom}(f(x))= \{ x / \forall x \in \mathbb{R}, \exists f(x) \}
    \end{split}
\end{equation}
\subsection{Asíntotas}
Una asíntota es una curva de una recta a la cual tiende la función en el
infinito de $x$ o de $f(x)$.
\subsubsection{Verticales}
Una recta en $x=x_{0}$ es una asíntota si se cumple que:
\begin{equation}
    \begin{split}
        \lim_{x \to x_{0}}f(x)&= \pm \infty\\
        \lim_{x \to x_{0}^{+}} f(x) &= \pm \infty\\
        \lim_{x \to x_{0}^{-}} f(x) &= \pm \infty
    \end{split}
\end{equation}
\subsubsection{Horizontales}
Una función tiene una asíntota horizontal si se cumple una de estas
condiciones:
\begin{equation}
    \begin{split}
        \exists \lim_{x \to \infty}f(x)\\
        \exists \lim_{x \to -\infty}f(x)
    \end{split}
\end{equation}
\subsubsection{Oblicuas}
Una asíntota oblicua es una recta de la forma $y=mx+n\: / \: m \neq 0$ a
la que $f(x)$ se aproxima en $\pm \infty$.
\begin{equation}
    \begin{split}
        \lim_{x \to \infty} f(x) = mx+n 
    \end{split}
\end{equation}
Para obtener la pendiente, podemos despejar $m$ para obtener:
\begin{equation}
    \begin{split}
        m = \lim_{x \to \infty} \frac{f(x)}{x}
    \end{split}
\end{equation}
Similarmente, podemos reordenar $n$ para obtener:
\begin{equation}
    \begin{split}
        n = \lim_{x \to \infty} f(x) -mx
    \end{split}
\end{equation}
\subsection{Monotonía}
Una función es creciente en un intervalo $[a,b]$ para cualquier par de
puntos $x_{1},x_{2} \in [a,b] / x_{1} < x_{2} \implies  f(x_{1}) < f(x_{2})$.\\
Si $f(x)$ es derivable en $(a,b)$:
\begin{itemize}
    \item $f'(x) > 0 \forall x \in  (a,b) \implies f(x)$ es creciente en $(a,b)$.
    \item $f'(x) < 0 \forall x \in  (a,b) \implies  f(X)$ es decreciente en $k(a,b)$.
    \item $f'(x) = 0 \forall  x \in (a,b) \implies f(x)$ es constante en $(a,b)$.
\end{itemize}
\subsubsection{Puntos críticos}
Una función derivable en $x=a$ tiene un máximo o un mínimo si $f'(a)=0$.\\
Un punto crítico es donde la derivada es $0$ o no existe.
\subsection{Curvatura}
Una función $f(x)$ es cóncava hacia arriba en el intervalo $(a,b)$ si en
todos los puntos del intervalo, la curva está por encima de las tangentes
en dichos puntos.
\begin{proof}[Definición]
    Llamamos \textbf{puntos de inflexión} de una curva a los puntos donde
    cambia la concavidad. Son los puntos donde la recta corta a la función.
    Debemos comprobar si la segunda derivada se anula en dicho punto, y si
    cambia de signo.
\end{proof}
\begin{proof}[Definición]
    Sea $f: \mathbb{R} \to \mathbb{R}$, diremos que es cóncava hacia arriba
    en $(a,b)$ si
    \begin{equation}
        \begin{split}
            \forall x \in (a,b) \exists \varepsilon > 0 / 
            f'(x+ \varepsilon) > f'(x)
        \end{split}
    \end{equation}
\end{proof}
\begin{derivada_par_impar}[Teorema]
    Sea $f: [a,b] \to \mathbb{R}$ una $k+1$ derivable en $(a,b)$.\\
    Sea $c \in (a,b)$, si $f'(c)=f'(c)=\dots=f^k(c)=0$ y $f^{k+1}(c) \neq 0$:
    \begin{itemize}
        \item Si $k$ es impar, $f(c)$ es un máximo relativo.
        \item Si $k$ es par, $f(c)$ es un punto de inflexión.
    \end{itemize}
    
\end{derivada_par_impar}
\section{Recta tangente y normal}
Para hallar la ecuación de la recta tangente de una función en un punto $a$, se debe cumplir que:
\begin{equation}
    \begin{split}
        \dv{y}{x}=f'(a)\\
        y(a)=f(a)
    \end{split}
\end{equation}
Con esto podemos llegar a que la recta se describe por la siguiente ecuación:
\begin{equation}
    \begin{split}
        y=f(a)+f'(a)(x-a)
    \end{split}
\end{equation}
Para obtener la recta normal, debemos transformar la pendiente para que pase a ser $m \to - \frac{1}{m}$:
\begin{equation}
    \begin{split}
        y= f(a) -\frac{1}{f'(a)}(x-a)
    \end{split}
\end{equation} 
\section{Tipos de funciones}
\begin{itemize}
    \item Injectiva: a cada valor del dominio de la función, le corresponde un valor de su rango.
    Para estas funciones, se cumple que:
    \begin{equation}
        \begin{split}
            f(a) = f(b) \implies a = b
        \end{split}
    \end{equation}
    Ejemplos: $2x, 5x+5, x^3 + 2x$
    \item Biyectiva: a todos los elementos del rango, les corresponde un \textbf{único}
    valor de la función. Es decir, no hay ningún elemento del dominio que comparta valor en
    la imagen.
    \item Sobreyectiva: múltiples valores del dominio pueden corresponder a un único valor de su rango.
\end{itemize}
\subsection{Funciones periódicas}
Una función $f$ es periódica de período $T$, si $\forall x \in Dom(f)$:
\begin{equation}
    \begin{split}
        f(x)=f(x+ kT)\: / \: k \in  \mathbb{Z}
    \end{split}
\end{equation}
\section{Funciones inversas}
La inversa de una función $f(x)$ es la que cumple que:
\begin{equation}
    \begin{split}
        (fof^{-1})(x)=x
    \end{split}
\end{equation}
La derivada de una función inversa es:
\begin{equation}
    \begin{split}
        \dv{f^{-1}(y)}{y}= \frac{1}{(fof^{-1}(y))}
    \end{split}
\end{equation}
\begin{rolle}
    Sea una función $f: \mathbb{R} \to \mathbb{R}$ continua en $[a,b]$ y derivable en $(a,b)$, si
    $f(a)=f(b)\Rightarrow \exists c \in (a,b): f'(c)=0$. 
\end{rolle}
\begin{proof}[Demostración]
    Para demostrarlo, vamos a partir de que toda función contínua en un intervalo cerrado y acotado
    alcanza un valor máximo y un valor mínimo absolutos en dicho intervalo.\\
    Como el máximo y el mínimo son análogos, vamos a demostrar para el máximo.\\
    \textbf{Caso 1}: el máximo no pertenece a $(a,b)$, que implica que $Max=f(a)$ o $Max=f(b)$. Si
    $f(a)=f(b)\Rightarrow Max=Min \Rightarrow f(x)=cte\Rightarrow f'(x)=0$.
    \textbf{Caso 2}: Supongamos que $m\in (a,b)$ es el máximo. Como $f(x)$ es derivable $\Rightarrow \exists f'(m)$.
    Como $m$ es un punto máximo, $f(x)$ es creciente en $a<x<m$:
    \begin{equation}
        \begin{split}
            \lim_{x \to m^-} \frac{f(x)-f(m)}{x-m}\geq 0\\
            \lim_{x \to m^+} \frac{f(x)-f(m)}{x-m}\leq 0
        \end{split}
    \end{equation}
    Al ser derivable, ambos límites deben coincidir, por lo que si se debe cumplir que
    \begin{equation}
        \begin{split}
            f'(m)\geq 0\\
            f'(m)\leq 0
        \end{split}
    \end{equation}
    Entonces
    \begin{equation}
        \begin{split}
            f'(m)=0
        \end{split}
    \end{equation}
\end{proof}
\begin{midval}
    Sean dos funciones $f,g : f: \mathbb{R} \to \mathbb{R} \wedge g: \mathbb{R} \to \mathbb{R}$, continuas y derivables en $[a,b]$.
    \begin{equation}
        \begin{split}
            \exists c \in (a,b): (f(b)-f(a))g'(c) = (g(b) - g(a))f'(c)
        \end{split}
    \end{equation}
\end{midval}
\begin{proof}[Demostración]
    Sea una función real y continua en $[a,b]$
    \begin{equation}
        \begin{split}
            h(x)= (f(b)-f(a))g(x)- (g(b)-g(a))f(x)
        \end{split}
    \end{equation}
    Si evaluamos $h(a)$ tenemos que
    \begin{equation}
        \begin{split}
            h(a)= f(b)g(a)-g(b)f(a)
        \end{split}
    \end{equation}
    Y $h(b)$ es
    \begin{equation}
        \begin{split}
            h(b)=f(b)g(a)-g(b)f(a)
        \end{split}
    \end{equation}
    Por tanto,
    \begin{equation}
        \begin{split}
            h(a)=h(b)\Rightarrow \exists c \in [a,b] : h'(c)=0
        \end{split}
    \end{equation}
    Es decir
    \begin{equation}
        \begin{split}
            h'(c)=(f(b)-f(a))g'(c)- (g(b)-g(a))f'(c)=0
        \end{split}
    \end{equation}
    Que implica que
    \begin{equation}
        \begin{split}
            (f(b)-f(a))g'(c)=(g(b)-g(a))f'(c)
        \end{split}
    \end{equation}
    QED
\end{proof}
\section{Álgebra de funciones}
\begin{equation}
    \begin{split}
        (f_1+f_2)(x)=f_1(x)+f_2(x)
    \end{split}
    \quad\quad
    \begin{split}
        (f_1 \cdot f_2)(x)=f_1(x) \cdot f_2(x)
    \end{split}
\end{equation}
Distinguimos entre funciones \textbf{elementales}, que son aquellas basadas en combinaciones de series o productos infinitos,
integrales indefinidas, a trozos o ecuaciones diferenciales.
\end{document}
