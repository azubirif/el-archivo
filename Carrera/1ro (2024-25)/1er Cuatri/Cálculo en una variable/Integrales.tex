\documentclass{article}
\author{Alejandro Zubiri}
\title{Integrales}

\renewcommand*\contentsname{Índice}

\usepackage[margin=1.1in]{geometry}
\usepackage{amsmath, physics, amsthm, amsfonts}

\newtheorem{teorema}{Teorema}
\newtheorem{defin}{Definición}

\newcommand{\R}{\mathbb{R}}

\begin{document}
\maketitle
\tableofcontents
\pagebreak
\section{Método de Hermite}
Supongamos que tenemos una integral de la forma:
\begin{equation}
	\begin{split}
		\int \frac{p(x)}{q(x)}
	\end{split}
\end{equation}
Queremos llegar a lo siguiente
\begin{equation}
	\begin{split}
		\int \frac{p(x)}{q(x)} = \frac{p_{1}(x)}{q_{1}(x)}=
		\int \frac{p_{2}(x)}{q_{2}(x)} \dd{x}
	\end{split}
\end{equation}
Siendo $q_{1}(x) = mcd(q(x), q'(x))$ y $q_{2}(x)= \frac{q(x)}{q_{1}(x)}$.
Luego, derivaremos la expresión entera y tendremos un sistema de ecuaciones que
podremos resolver.
\section{Aplicación del teorema fundamental del cálculo}
Teniendo que:
\begin{equation}
	\begin{split}
		F(x) = \int_{h(x)}^{g(x)} f(t) dt
	\end{split}
\end{equation}
Entonces:
\begin{equation}
	\begin{split}
		F'(x) = f(g(x))g'(x) - f(h(x))h'(x)
	\end{split}
\end{equation}
\end{document}
