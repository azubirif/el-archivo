\documentclass{article}
\author{Alejandro Zubiri}
\date{Wed Nov 27 2024}
\title{Sucesiones}

\usepackage{amsmath, amsthm}

\begin{document}
\maketitle
\tableofcontents
\pagebreak

\section{Sucesiones telescópicas}
El objetivo de una sucesión es encontrar el término $a_{n}$ en función de $n$ y de $a_{0}$.
Para resolver una sucesión, iremos sumando a ambos lados la ecuación inicial, restando
cada vez $1$ al valor de $n$, hasta que podamos expresar uno de los lados como un sumatorio:
\begin{equation}
    \begin{split}
        a_{n+1} - a_{n} &=  - \frac{a_{1}}{n^{2}}\\
        (a_{2}-a_{1}) \dots(a_{n}- a_{n-1})+(a_{n+1}-a_{n})&= -\frac{a_{1}}{n^{2}}-\frac{a_{1}}{(n-1)^{2}}
        - \dots - \frac{a_{1}}{1^{2}}
    \end{split}
\end{equation}
Del LI se mantienen $a_{n+1}-a_{1}$, y el LD se puede expresar como
$-\sum _{i=1}^n \frac{a_{1}}{n^{2}}$, obteniendo así:
\begin{equation}
    \begin{split}
        a_{k} = a_{1}(1-\sum _{i=1}^{k-1} \frac{1}{n^{2}})
    \end{split}
\end{equation}
\end{document}