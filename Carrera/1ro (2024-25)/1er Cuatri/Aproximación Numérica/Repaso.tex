\documentclass{article}
\author{Alejandro Zubiri}
\date{Wed Oct 09 2024}
\title{Repaso}

\usepackage{amsmath}
\usepackage{amsthm}
\usepackage{physics}
\usepackage{amsfonts}

\newtheorem{rolle}{Teorema de Rolle}[section]
\newtheorem{midval}{Teorema del punto medio}

\begin{document}
\maketitle
\section{Tipos de funciones}
\begin{itemize}
    \item Injectiva: a cada valor del dominio de la función, le corresponde un valor de su rango.
    \item Biyectiva: a todos los elementos del rango, les corresponde un valor de la función.
    \item Sobreyectiva: múltiples valores del dominio pueden corresponder a un único valor de su rango.
\end{itemize}
\section{Funciones inversas}
La inversa de una función $f(x)$ es la que cumple que:
\begin{equation}
    \begin{split}
        (fof^{-1})(x)=x
    \end{split}
\end{equation}
La derivada de una función inversa es:
\begin{equation}
    \begin{split}
        \dv{f^{-1}(y)}{y}= \frac{1}{(fof^{-1}(y))}
    \end{split}
\end{equation}
\begin{rolle}
    Sea una función $f: \mathbb{R} \to \mathbb{R}$ continua en $[a,b]$ y derivable en $(a,b)$, si
    $f(a)=f(b)\Rightarrow \exists c \in (a,b): f'(c)=0$. 
\end{rolle}
\begin{proof}[Demostración]
    Para demostrarlo, vamos a partir de que toda función contínua en un intervalo cerrado y acotado
    alcanza un valor máximo y un valor mínimo absolutos en dicho intervalo.\\
    Como el máximo y el mínimo son análogos, vamos a demostrar para el máximo.\\
    \textbf{Caso 1}: el máximo no pertenece a $(a,b)$, que implica que $Max=f(a)$ o $Max=f(b)$. Si
    $f(a)=f(b)\Rightarrow Max=Min \Rightarrow f(x)=cte\Rightarrow f'(x)=0$.
    \textbf{Caso 2}: Supongamos que $m\in (a,b)$ es el máximo. Como $f(x)$ es derivable $\Rightarrow \exists f'(m)$.
    Como $m$ es un punto máximo, $f(x)$ es creciente en $a<x<m$:
    \begin{equation}
        \begin{split}
            \lim_{x \to m^-} \frac{f(x)-f(m)}{x-m}\geq 0\\
            \lim_{x \to m^+} \frac{f(x)-f(m)}{x-m}\leq 0
        \end{split}
    \end{equation}
    Al ser derivable, ambos límites deben coincidir, por lo que si se debe cumplir que
    \begin{equation}
        \begin{split}
            f'(m)\geq 0\\
            f'(m)\leq 0
        \end{split}
    \end{equation}
    Entonces
    \begin{equation}
        \begin{split}
            f'(m)=0
        \end{split}
    \end{equation}
\end{proof}
\begin{midval}
    Sean dos funciones $f,g : f: \mathbb{R} \to \mathbb{R} \wedge g: \mathbb{R} \to \mathbb{R}$, continuas y derivables en $[a,b]$.
    \begin{equation}
        \begin{split}
            \exists c \in (a,b): (f(b)-f(a))g'(c) = (g(b) - g(a))f'(c)
        \end{split}
    \end{equation}
\end{midval}
\begin{proof}[Demostración]
    Sea una función real y continua en $[a,b]$
    \begin{equation}
        \begin{split}
            h(x)= (f(b)-f(a))g(x)- (g(b)-g(a))f(x)
        \end{split}
    \end{equation}
    Si evaluamos $h(a)$ tenemos que
    \begin{equation}
        \begin{split}
            h(a)= f(b)g(a)-g(b)f(a)
        \end{split}
    \end{equation}
    Y $h(b)$ es
    \begin{equation}
        \begin{split}
            h(b)=f(b)g(a)-g(b)f(a)
        \end{split}
    \end{equation}
    Por tanto,
    \begin{equation}
        \begin{split}
            h(a)=h(b)\Rightarrow \exists c \in [a,b] : h'(c)=0
        \end{split}
    \end{equation}
    Es decir
    \begin{equation}
        \begin{split}
            h'(c)=(f(b)-f(a))g'(c)- (g(b)-g(a))f'(c)=0
        \end{split}
    \end{equation}
    Que implica que
    \begin{equation}
        \begin{split}
            (f(b)-f(a))g'(c)=(g(b)-g(a))f'(c)
        \end{split}
    \end{equation}
    QED
\end{proof}
\end{document}
