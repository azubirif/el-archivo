\documentclass{article}
\title{Teoremas Final}
\author{Alejandro Zubiri}

\usepackage{physics, amsmath, amsthm, amsfonts}

\newtheorem{theorem}{Teorema}

\begin{document}
\maketitle
\section{Teorema Fundamental del Cálculo Integral}
    Sea $f$ una función continua en $[a,b]$, y sea $F(x)= \int _{a}^x f(t) \dd{t} / x \in [a,b] \implies $ F es
    derivable y
    \begin{equation}
        \begin{split}
            F'(x)=f(x) \forall x \in [a,b]
        \end{split}
    \end{equation}
\begin{proof}[Demostración]
    \begin{equation}
        \begin{split}
            F'(x) = \lim_{h \to 0} \frac{F(x+h) - F(x)}{h} = \lim_{h \to 0} \frac{ \int _{x}^{x+h} f(t) \dd{t}}{h}
        \end{split}
    \end{equation}
    Ahora podemos aplicar el teorema del valor medio para integrales:
    \begin{equation}
        \begin{split}
            \exists c \in [x,x+h] / f(c) \cdot h = \int _{x}^{x+h}f(t) \dd{t}
        \end{split}
    \end{equation}
    Por tanto
    \begin{equation}
        \begin{split}
            F'(x)= \frac{1}{h} \int _{x}^{x+h} f(t) \dd{t} = \lim_{h \to 0} \frac{f(c) \cdot h}{h}=\lim_{h \to 0}f(c)
            =f(x)
        \end{split}
    \end{equation}
Para memorizar:
\begin{itemize}
	\item Definir \(F(x) = \int_a^x f(t) \dd{t}\).
	\item Aplicar derivada por definición.
	\item Usar el TVM para integrales.
\end{itemize}
\end{proof}
\section{Teorema del valor medio}
    Sean dos funciones $f,g : f: \mathbb{R} \to \mathbb{R} \wedge g: \mathbb{R} \to \mathbb{R}$, continuas y derivables en $[a,b]$.
    \begin{equation}
        \begin{split}
            \exists c \in (a,b): (f(b)-f(a))g'(c) = (g(b) - g(a))f'(c)
        \end{split}
    \end{equation}
\begin{proof}[Demostración]
    Sea una función real y continua en $[a,b]$
    \begin{equation}
        \begin{split}
            h(x)= (f(b)-f(a))g(x)- (g(b)-g(a))f(x)
        \end{split}
    \end{equation}
    Si evaluamos $h(a)$ tenemos que
    \begin{equation}
        \begin{split}
            h(a)= f(b)g(a)-g(b)f(a)
        \end{split}
    \end{equation}
    Y $h(b)$ es
    \begin{equation}
        \begin{split}
            h(b)=f(b)g(a)-g(b)f(a)
        \end{split}
    \end{equation}
    Por tanto,
    \begin{equation}
        \begin{split}
            h(a)=h(b)\Rightarrow \exists c \in [a,b] : h'(c)=0
        \end{split}
    \end{equation}
    Es decir
    \begin{equation}
        \begin{split}
            h'(c)=(f(b)-f(a))g'(c)- (g(b)-g(a))f'(c)=0
        \end{split}
    \end{equation}
    Que implica que
    \begin{equation}
        \begin{split}
            (f(b)-f(a))g'(c)=(g(b)-g(a))f'(c)
        \end{split}
    \end{equation}
    QED\\
    Para memorizar:
    \begin{itemize}
	    \item La función $h(x) = [f(b)-f(a)]g(x) - [g(b)-g(a)]f(x)$
		\item Aplicar teorema de Rolle.
	\end{itemize}
\end{proof}
\section{Teorema de Rolle}
    Sea una función $f: \mathbb{R} \to \mathbb{R}$ continua en $[a,b]$ y derivable en $(a,b)$, si
    $f(a)=f(b)\Rightarrow \exists c \in (a,b): f'(c)=0$.
\begin{proof}[Demostración]
    Para demostrarlo, vamos a partir de que toda función contínua en un intervalo cerrado y acotado
    alcanza un valor máximo y un valor mínimo absolutos en dicho intervalo.\\
    Como el máximo y el mínimo son análogos, vamos a demostrar para el máximo.\\
    \textbf{Caso 1}: el máximo no pertenece a $(a,b)$, que implica que $Max=f(a)$ o $Max=f(b)$. Si
    $f(a)=f(b)\Rightarrow Max=Min \Rightarrow f(x)=cte\Rightarrow f'(x)=0$.
    \textbf{Caso 2}: Supongamos que $m\in (a,b)$ es el máximo. Como $f(x)$ es derivable $\Rightarrow \exists f'(m)$.
    Como $m$ es un punto máximo, $f(x)$ es creciente en $a<x<m$:
    \begin{equation}
        \begin{split}
            \lim_{x \to m^-} \frac{f(x)-f(m)}{x-m}\geq 0\\
            \lim_{x \to m^+} \frac{f(x)-f(m)}{x-m}\leq 0
        \end{split}
    \end{equation}
    Al ser derivable, ambos límites deben coincidir, por lo que si se debe cumplir que
    \begin{equation}
        \begin{split}
            f'(m)\geq 0\\
            f'(m)\leq 0
        \end{split}
    \end{equation}
    Entonces
    \begin{equation}
        \begin{split}
            f'(m)=0
        \end{split}
    \end{equation}
Para memorizar demostración:
\begin{itemize}
	\item Hacer casos para los cuales \(Max \in (a,b)\) y los que no.
	\item Hacer derivada por definición de \(f'(max)\) y hacer \(\leq\) y \(\geq\).  
\end{itemize}
\end{proof}
\section{Teorema del error del método de iteración de punto fijo}
    Sea la iteración definida para el método de iteración de punto fijo donde $g(x)$ cumple las
    hipótesis del teorema del punto fijo para un intervalo $[a,b]$.\\
    Sea un $x_{0} \in [a,b]$, sea $\varepsilon > 0$ la tolerancia y sea $k_{0}(\varepsilon )$
    el entero positivo más pequeño tal que $|x_{k} - \xi | < \varepsilon  \forall k \geq k_{0}(\varepsilon )$:
    \begin{equation}
        \begin{split}
            K_{0}(\varepsilon ) \leq \left \lfloor{\frac{\ln(x_{1}-x_{0})-\ln (\varepsilon (1-L))}
            {\ln (\frac{1}{L})}}\right \rfloor +1
        \end{split}
    \end{equation}
\begin{proof}[Demostración]
    Primero, demostramos por inducción que:
    \begin{equation}
        \begin{split}
            |x_{k}-\xi | \leq L^{K} |x_{0}-\xi | \forall k \geq 1
        \end{split}
    \end{equation}
    Empezamos por el caso base:
    \begin{equation}
        \begin{split}
            |x_{1}-\xi | \leq L|x_{0}-\xi |
        \end{split}
    \end{equation}
    Tal que:
    \begin{equation}
        \begin{split}
            x_{1}=g(x_{0}) \quad \xi =g(\xi )
        \end{split}
    \end{equation}
    \begin{equation}
        \begin{split}
            \lvert g(x_{0})-g(\xi ) \rvert \leq L \lvert x_{0}-\xi  \rvert  
        \end{split}
    \end{equation}
    Que se cumple $\forall x_{0} \in [a,b], L \in (0,1)$ ya que es contractiva.\\
    Asumimos para $n=p$:
    \begin{equation}
        \begin{split}
            \lvert x_{p}-\xi  \rvert \leq L^{p}|x_{0}-\xi |
        \end{split}
    \end{equation}
    Deberá cumplirse para $n=p+1$:
    \begin{equation}
        \begin{split}
            \lvert x_{p+1}-\xi  \rvert = \lvert g(x_{p})-g(\xi ) \rvert \leq L\lvert x_{p}-\xi \rvert
            \leq L^{p}L \lvert x_{0}-\xi  \rvert\leq L^{p+1}\lvert x_{0}-\xi  \rvert   
        \end{split}
    \end{equation}
    Volvemos para el caso $n=1$:
    \begin{equation}
        \begin{split}
            \lvert x_{0}-\xi  \rvert &= \lvert x_{0}-\xi +x_{1}-x_{1} \rvert\\
            &\leq \lvert x_{0}-x_{1} \rvert + \lvert x_{1}-\xi  \rvert\\
            &\leq \lvert x_{0}-x_{1} \rvert  +L\lvert x_{0}-\xi  \rvert
		\end{split}
	\end{equation}
	\begin{equation}
		\begin{split}
            \lvert x_{0}-\xi  \rvert &\leq \lvert x_{0}-x_{1} \rvert + L\lvert x_{0}-\xi  \rvert\\
            \lvert x_{0}-\xi \rvert (1-L) &\leq \lvert x_{0}-x_{1} \rvert \\ 
            \lvert x_{0}-\xi  \rvert &\leq \frac{\lvert x_{0}-x_{1} \rvert }{1-L}\\
            \lvert x_{k}-\xi  \rvert &\leq L^{K}\lvert x_{0}-\xi  \rvert \leq L^{k} \frac{1}{1-L}
            \lvert x_{1}-x_{0} \rvert   
        \end{split}
    \end{equation}
    Tenemos que:
    \begin{equation}
        \begin{split}
            \lvert x_{k}-\xi  \rvert &< \varepsilon \forall k \geq k_{0}\\
            \lvert x_{k}-\xi  \rvert&\leq \frac{L^{k}}{1-L}\lvert x_{1}-x_{0} \rvert<\varepsilon\\
			L^{k}\lvert x_{1}-x_{0} \rvert &\leq \varepsilon (1-L)
        \end{split}
    \end{equation}
    Luego despejamos para $k$:
    \begin{equation}
        \begin{split}
            k \ln L + \ln \lvert x_{1}-x_{0} \rvert &\leq \ln(\varepsilon (1-L))\\
            \ln \lvert x_{1}-x_{0} \rvert - \ln (\varepsilon (1-L)) &\leq -k \ln L = k \ln (\frac{1}{L})\\
			k &> \frac{\ln(x_{1}-x_{0})-\ln (\varepsilon (1-L))}{\ln (\frac{1}{L})}
        \end{split}
    \end{equation}
    Para que obtengamos los mismos valores y sea una cota superior:
    \begin{equation}
        \begin{split}
            K_{0}(\varepsilon ) \leq \left \lfloor{\frac{\ln(x_{1}-x_{0})-\ln (\varepsilon (1-L))}
            {\ln (\frac{1}{L})}}\right \rfloor +1
        \end{split}
    \end{equation}
	
\end{proof}
Para memorizar:
\begin{itemize}
	\item Demostrar por inducción que \(|x_{k}-\xi | \leq L^{K} |x_{0}-\xi | \forall k \geq 1\)
	\item Reordenar hasta obtener \(|x_0 - \xi |(1-L) \leq |x_0-x_1|\)
	\item Utilizar lo demostrado por induccion y sistituir \(|x_0-\xi|\)
	\item Aplicar desigualdad para \(|x_k - \xi|<\varepsilon\) y despejar para \(K\).
	\item Aplicar función suelo y \(+1\). 
\end{itemize}
\section{Teorema del error del método de interpolación polinómica}
	Sea $f$ con $n > 0$ derivadas continuas en $[a,b]/\exists f^{n+1)}(x)$ en $(a,b)$ y sean $x_0, x_1,\dots,x_n \in [a,b]$ y 
	$x_i \neq x_j$ y $P(x)$ el polinomio interpolador de esos puntos.
	Para cada $x \in [a,b] \exists \xi / min(x_0,\dots,x_n) < \xi < max(x_0,\dots,x_n)/$
	\begin{equation}
		\begin{split}
			f(x)-P(x) = \frac{(x-x_0)(x-x_1)\dots(x-x_n)}{(n+1)!}f^{n+1)}(\xi)
		\end{split}
	\end{equation}
\begin{proof}[Demostración]
	$x \in [a,b]$ si $\exists n \geq i \geq 0, x_{i} = x \implies $ se cumple siempre (ya que hemos definido el polinomio como tal).
	Si $x \neq x_{i} \forall i \in \mathbb{N} \implies
	M(x) = \frac{f(x)-P(x)}{(x-x_{0})\dots(x-x_{n)}}$\\
	Queremos saber si
	\begin{equation}
		\begin{split}
			\exists \xi \in [a,b] / f^{n+1)}(\xi)=M(n+1)!
		\end{split}
	\end{equation}
	Ya que entonces se cumplirá que:
	\begin{equation}
		\begin{split}
			f^{n+1)(\xi)}=(n+1)!\frac{f(x)-P(x)}{(x-x_{0})\dots(x-x_{n)}}
		\end{split}
	\end{equation}
	Que implica la proposición inicial.\\
	Sea $g(t)=f(t)-P(t) - M(t-x_{0})\dots (t-x_{n})$
	\begin{equation}
		\begin{split}
			g^{n+1)}=f^{n+1)}(t)-P^{n+1)}(t)-M(n+1)!
		\end{split}
	\end{equation}
	Como el grado de $P(t)$ es $\leq n$, esta derivada es $0$.
	\begin{equation}
		\begin{split}
			= f^{n+1)}-M(n+1)!
		\end{split}
	\end{equation}
	Ahora queremos saber si $\exists t \in[a,b] / g^{n+1)}(t)=0$\\
	Si $x=t$:
	\begin{equation}
		\begin{split}
			g(t)&=f(t)-P(t)- \frac{f(t)-P(t)}{(t-x_{0})\dots(t-x_{n})}
			(t-x_{0})\dots(t-x_{n})\\
			    &=0
		\end{split}
	\end{equation}
	Por otro lado, $g(t)=0 \implies x=x_{0},x_{n}$, es decir, los nodos.\\
	Por tanto, $g(t)$ tiene $n+2$ raíces distintas, y $g'(t)$ tiene $n+1$ raíces,
	así que $g^{n+1)}(t)$ tiene una raíz. Hemos llegado a la conclusión deseada. 
\end{proof}
Para memorizar:
\begin{itemize}
	\item Definir $M(x) = \frac{f(x)-P(x)}{(x-x_{1})\dots(x-x_{n})}$
	\item Plantear $\exists \xi \in [a,b] / f^{n+1)}(\xi)=M(n+1
    )!$
    	\item Definit $g(t)=f(t)-P(t) - M(t-x_{0})\dots (t-x_{n})$ y sacar derivada $n+1$.
	\item Intentar igualar la derivada anterior a $0$, y hacerlo mediante $t=x$ o $x=x_{0},\dots,x_{n}$ (nodos)
	\item Hacer lo de las raíces en función del número de la derivada, hasta llegar a la derivada $n+1$ y decir que tiene una raíz. 
\end{itemize}
\end{document}
