\documentclass{article}
\author{Alejandro Zubiri}
\date{Thu Nov 14 2024}
\title{Errores}

\usepackage{amsmath, amsthm, amsfonts}

\begin{document}
\maketitle
\tableofcontents
\pagebreak

\section{Redondeos}
Acción de reemplazar un número por otro con menos dígitos
\subsection{Reglas}
\begin{enumerate}
    \item Los $n$ primeros dígitos se mantienen y se descarta el resto. Rellenamos con ceros.
    \item Si los dígitos descartados forman un número menor a 5, entonces los dígitos
    conservados no cambian (redondeo por defecto). Si es mayor, el último de los dígitos
    conservados aumenta en 1.
    \item Si es igual a 5, no hay regla general, pero \textbf{nosotros} sumaremos 1.
\end{enumerate}
\section{Cálculo de error absoluto y relativo}
El error de redondeo es el que resulta al sustituir un cierto número $p$ por su forma
redondeada $\hat{p}$.\\
El error absoluto de aproximación a $p$: $E_{A}(\hat{p}) = \lvert p-\hat{p} \rvert $.
Error relativo de aproximación: $E_{r}(\hat{p}) = \frac{\lvert p-\hat{p} \rvert }{\lvert p \rvert }$.\\
Cuando no tenemos el valor real, aproximamos:
\begin{equation}
    \begin{split}
        E_{r}(\hat{p}) \approx \frac{E_{A}(\hat{p})}{\lvert \hat{p} \rvert }
    \end{split}
\end{equation}
El error relativo porcentual es:
\begin{equation}
    \begin{split}
        \epsilon _{r} ( \hat{p}) = E_{r}(\hat{p})\cdot 100 \%
    \end{split}
\end{equation}
\begin{proof}[Cifras significativas]
    Aquellas situadas a la derecha del primer dígito nu nulo.
\end{proof}
\begin{proof}[Definición]
    Sea $\hat{p}$ una aproximación de $p$:
    \begin{equation}
        \begin{split}
            \hat{p} = \pm (\alpha \cdot 10^{m}+ \alpha _{2} \cdot 10^{m-1}+ \dots)
        \end{split}
    \end{equation}
    $\alpha _{n}$ es una cifra significativa válida cuando el error absoluto sea:
    \begin{equation}
        \begin{split}
            E_{A} (\hat{p}) \leq 0.5\cdot 10^{m-n+1}
        \end{split}
    \end{equation}
    Podemos afirmar que $\alpha_{n}$ es CSV si:
    \begin{equation}\tag*{Utilizamos esta cuando vamos de $CSV$ a $E_{r}$}
        \begin{split}
            E_{r}(\hat{p}) \leq \frac{0.5}{(\alpha _{n} +1)\cdot 10^{n-1}}
        \end{split}
    \end{equation}
    Si $\alpha _{n}$ es la última CSV de $\hat{p}$, entonces:
    \begin{equation}
        \begin{split}
            E_{A}(\hat{p}) = 0.5\cdot 10^{m-n+1}
        \end{split}
    \end{equation}
\end{proof}
\section{Operaciones con errores}
\subsection{Suma y resta}
\begin{itemize}
    \item Localizar los números con mayor error absoluto
    \item Redondear los restantes, reteniendo un dígito más que en los redondeados anteriormente
    \item Sumamos o restamos los valores
    \item Redondeamos y descartamos el último obtenido
    \item Tomamos como error absoluto la suma de errores absolutos de los números menos exactos
    más el error absoluto del redondeo
    \item Sacamos $E_{r}$ con la fórmula
    \item Sacamos el número de CSVs con $E_{A}$
\end{itemize}
\subsection{Multiplicación}
\begin{itemize}
    \item Localizar los números con menos CSVs
    \item Redondear el resto, reteniendo uno más
    \item Operamos con los valores sin errores
    \item Redondear el resultado, reteniendo tantos dígitos como cifras exactas había en el
    operando menos exacto
    \item $E_{r}$ con la suma de errores relativos
    \item $E_{A}$ a partir de error relativo
    \item $E_{A}$ para hallar los CSVs
\end{itemize}

\end{document}