\documentclass{article}
\author{Alejandro Zubiri}
\date{Wed Dec 11 2024}
\title{Aplicaciones Lineales}

\usepackage{amsmath, amsfonts}

\newtheorem{defin}{Definición}
\newtheorem{teor}{Teorema}

\begin{document}
\maketitle
\tableofcontents
\pagebreak
Una correspondencia entre dos conjuntos:\\
A cada elemento de $A$ le corresponde un único elemento de $B$.\\
\begin{defin}
    Una aplicación entre dos EV es lineal si:
    \begin{equation}
        \begin{split}
            f(ku + \alpha v) = kf(u) + \alpha f(v)
        \end{split}
    \end{equation}
    Además, se cumplirá que:
    \begin{equation}
        \begin{split}
            f(\vec{0}) = f(0\cdot \bar{u}) = 0f(\bar{u})= \bar{0}
        \end{split}
    \end{equation}
\end{defin}
\section{Espacio Dual}
Conjunto de aplicaciones de un EV a un cuerpo $\mathbb{K}$:
\begin{equation}
    \begin{split}
        V^{*} = \{ w / V \to  \mathbb{K}, w \text{ lineal} \}
    \end{split}
\end{equation}
Una A.L. siempre se puede expresar de forma matricial:
\begin{equation}
    \begin{split}
        Y = AX
    \end{split}
\end{equation}
Donde $Y$ son las coordenadas finales, $X$ las coordenadas originales, y $A$ la matriz asociada
a la A.L..\\
$A$ es de orden $m\times n$, y son las imágenes de los vectores de la base canónica.\\
Si $B = \{ u_{1},\dots,u_{n} \}$, entonces:
\begin{equation}
    \begin{split}
        A = (f(u_{1}), \dots,f(u_{n}))
    \end{split}
\end{equation}
\section{Endomorfismo}
Una aplicación lineal de un espacio sobre sí mismo.
\begin{teor}
    Sea $B_{c}$ la base canónica y $B'$ otra base. Sea $Y=AX$ la expresión de la aplicación respecto
    a la base canónica y $Y'=A'X'$ respecto a $B'$
    \begin{equation}
        \begin{split}
            A^{-1}=C^{-1}A = [f(B')]_{B'}
        \end{split}
    \end{equation}
    Donde $C$ es la matriz de cambio de base de $B' \to B_{C}$. Esto es como si pasásemos
    de $B'$ a $B_{c}$, aplicásemos la aplicación, y luego pusiésemos las coordenadas respecto
    a la base $B'$.
\end{teor}
\section{Diagonalización de endomorfismos}
Como $A'=C^{-1}AC$, se dice que $A'$ y $A$ son matrices semejantes con matriz de paso $C$. Dos matrices
asociadas a un endomofrismo siempre son semejantes.\\
Para diagonalizar un endomorfismo, buscamos la base respecto a la cual la matriz del endomorfismo es
\textbf{diagonal}. Esta base estará formada por los \textbf{autovectores} de la aplicación.
\begin{defin}
    Los autovectores $\bar{u}_{i}$ de una aplicación son aquellos que cumplen que:
    \begin{equation}
        \begin{split}
            A \bar{u}_{i} = \lambda_{i} \bar{u}_i / \lambda_{i} \in \mathbb{R}
        \end{split}
    \end{equation}
\end{defin}
Los coeficientes $\lambda _{i}$ que multiplican a los autovectores son los \textbf{autovalores}.\\
Además, se cumple que:
\begin{itemize}
    \item Si $\bar{u}$ es un autovector de un autovalor, $\alpha \bar{u}$ también lo es.
\end{itemize}
Para buscar autovalores, buscamos aquellos que cumplan:
\begin{equation}
    \begin{split}
        A \bar{v} &= \lambda \bar{v}\\
        (A - I\lambda ) \vec{v} &= \vec{0}
    \end{split}
\end{equation}
Las soluciones $\vec{v}$ de este sistema son el núcleo. Este forma un subespacio vectorial, denominado
\textbf{autoespacio de $\lambda $}.
\begin{defin}
    El autoespacio de un autovalor $\lambda $ es el subespacio de vectores que cumplen que:
    \begin{equation}
        \begin{split}
            (A - I\lambda ) \vec{v} &= \vec{0}
        \end{split}
    \end{equation}
    Para que esto se cumpla para infinitos vectores, el sistema debe ser SCI, y por tanto:
    \begin{equation}
        \begin{split}
            \boxed{\det (A - I\lambda) = 0}
        \end{split}
    \end{equation}
\end{defin}
\begin{defin}
    El número de veces que aparece un autovalor en la ecuación característica es su
    \textbf{multiplicidad algebraica} (m.a.).
\end{defin}
\begin{defin}
    La dimensión del autoespacio de cada autovalor $H_{\lambda }$ es la \textbf{multiplicidad geométrica}
    de dicho autovalor.
\end{defin}
Para cualquier aplicación, se cumple que:
\begin{equation}
    \begin{split}
        \boxed{1 \le m.g. \le m.a. \le n}
    \end{split}
\end{equation}
Donde $n$ es la dimensión del espacio del endomorfismo.\\
Para que un endomorfismo sea diagonalizable, se debe cumplir que:
\begin{equation}
    \begin{split}
        m.g. = m.a.
    \end{split}
\end{equation}
Podemos unir ambas propiedades para obtener que:
\begin{itemize}
    \item Si un autovalor tiene $m.a. = 1 \implies m.g. = 1$
    \item Si todos los autovalores son simples $(m.a.=1)$, el endomorfismo es diagonalizable.
    \item La matriz que representa la aplicación estará formada por los autovalores, en el mismo orden que estén los
    autovectores en la base elegida.
\end{itemize}
\end{document}