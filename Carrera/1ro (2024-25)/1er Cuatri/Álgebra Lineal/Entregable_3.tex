\documentclass{article}
\title{Entregable 3}
\author{Alejandro Zubiri}

\usepackage{amsmath, amsfonts, amsthm, enumitem}

\begin{document}
\maketitle
Sea la aplicación lineal definida por \(f: \mathbb{R}^{3} \to \mathbb{R}^{3} / 
f(x,y,z) = (3x-y, 2y-x-z, 3z-y)\) 
\begin{enumerate}[label={(\alph*)}]
	\item Para que sea un endomorfismo, el espacio vectorial inicial y final deben coincidir,
		y la aplicación debe ser lineal.\\Podemos ver que ambos espacios vectoriales coinciden,
		por lo que falta ver si la aplicación es lineal. Para que lo sea, se debe cumplir que:
		\begin{itemize}
			\item \(f(a \vec{u})=af( \vec{u}) /a \in \mathbb{R}, \vec{u} \in \mathbb{R}^{3}\)\\
				\begin{equation}
					\begin{split}
						f(a \vec{u}))&=(3au_1-au_2,2au_2-au_1-au_3,3au_3-au_2)\\
									 &=(a(3u_1-u-2),a(2u_2-u_1-u_3),a(3u_3-u_2))\\
									 &=af(\vec{u})
					\end{split}
				\end{equation}
			\item \(f(\vec{u}+\vec{e})=f(\vec{u})+f(\vec{e}) / \vec{e} \in \mathbb{R}^{3}\)
				\begin{equation}
					\begin{split}
				f(\vec{u}+\vec{e})&=(3u_1+3e_1-u_2-e_2,2u_2+2e_2-u_1-e_1-u_3-e_3,3u_3+3e_3-u_2-e_2)\\
				&= f(\vec{u})+f(\vec{e})
					\end{split}
				\end{equation}
		\end{itemize}
	Para encontrar la matriz asociada al endomorfismo:
	\begin{equation}
		\begin{split}
			(3x-y,2y-x-z,3z-y)&=\\(3x,-x,0)+(-y,2y,-y)+(0,-z,3z)&=\\ \begin{pmatrix}
3 & -1 & 0 \\
-1 & 2 & -1 \\
0 & -1 & 3 \\
\end{pmatrix}\begin{pmatrix}
x \\
y \\
z\end{pmatrix}
		\end{split}
	\end{equation}
	Por tanto, la matriz asociada respecto a la base canónica es:
	\begin{equation}
		\begin{split}
			A_{B_c}=  \begin{pmatrix}
3 & -1 & 0 \\
-1 & 2 & -1 \\
0 & -1 & 3 \\
\end{pmatrix}
		\end{split}
	\end{equation}
\item Teniendo el vector \(\vec{v}=(1,2,0)\), su imagen es:
	\begin{equation}
		\begin{split}
			f(\vec{v})= \begin{pmatrix}
3 & -1 & 0 \\
-1 & 2 & -1 \\
0 & -1 & 3 \\
\end{pmatrix}\begin{pmatrix}
1 \\
2 \\
0\end{pmatrix}=(1,3,-2)
		\end{split}
	\end{equation}
\item Ahora siendo \(\vec{w}=(1,1,1)\), buscamos si \(\exists \vec{u} \in \mathbb{R}^{3} /
	f(\vec{u})=\vec{w}\). Para hacerlo, igualamos cada componente de la aplicación a cada
	componente del vector:
	\begin{equation}
		\begin{split}
			3x-y&=1\\-x+2y-z&=1\\-y+3z&=1
		\end{split}
	\end{equation}
	Que tiene como solución:
	\begin{equation}
		\begin{split}
			\vec{u}=(0.75,1.25,0.75)
		\end{split}
	\end{equation}
\item Para encontrar la matriz asociada respecto a la base
	\begin{equation}
		\begin{split}
			B' = \{ (1,0,1),(1,-1,0),(1,0,0) \} = \{ v_1,v_2,v_3 \}
		\end{split}
	\end{equation}
	Queremos las coordenadas de las imágenes de cada vector de la base respecto a la base:
	\begin{equation}
		\begin{split}
			f(v_1)&=(3,-2,3)\\f(v_2)&=(4,-3,1)\\f(v_3)&=(3,-1,0)
		\end{split}
	\end{equation}
	Y sus respectivas coordenadas son:
	\begin{equation}
		\begin{split}
			[f(v_1)]_{B'} &= (3,2,-2)\\
			[f(v_2)]_{B'} &= (1,3,0)\\
			[f(v_3)]_{B'} &= (0,1,2)
		\end{split}
	\end{equation}
	Por tanto, la matriz es:
		\begin{equation}
			\begin{split}
				A_{B'}=\begin{pmatrix}
3 & 1 & 0 \\
2 & 3 & 1 \\
-2 & 0 & 2 \\
\end{pmatrix}
			\end{split}
		\end{equation}
	\item Siendo \(\vec{z}=(0,1,-1)\), para encontrar las coordenadas de la imagen respecto a
		\(B'\), simplemente la multiplicamos por la matriz asociada respecto a \(B'\):
		\begin{equation}
			\begin{split}
				[f(\vec{z})]_{B'} = A_{B'} \vec{z}=(-1,-4,-2)
			\end{split}
		\end{equation}
	\item Primero debemos buscar los autovalores, por lo que tenemos que resolver la ecuación:
		\begin{equation}
			\begin{split}
				\det (A-\lambda I)= 0
			\end{split}
		\end{equation}
		Donde \(A\) es la matriz asociada, \(\lambda\) es cada autovector, y \(I\) es la matriz
		identidad.
\begin{equation}
	\begin{split}
		\begin{vmatrix}
3-\lambda & -1 & 0 \\
-1 & 2-\lambda  & -1 \\
0 & -1 & 3-\lambda \\
\end{vmatrix}=(3-\lambda)^{2}(2-\lambda)-2(3-\lambda)=-(\lambda -3)(\lambda -1)(\lambda -4)
	\end{split}
\end{equation}
Ahora lo igualamos a cero:
\begin{equation}
	\begin{split}
		(\lambda -3)(\lambda -1)(\lambda-4)=0 \implies \lambda_1=3,\lambda_2=1,\lambda_3=4	
	\end{split}
\end{equation}
Puesto que la \(m.a.\) de cada autovalor es \(1\), sabemos que su \(m.g.\) también será \(1\),
y por tanto el endomorfismo es diagonalizable.\\
Para encontrar la base, buscamos los autovectores reemplazando cada autovalor en la ecuación
característica y utilizando un vector genérico \(\vec{v}=(x,y,z)\):
\begin{equation}
	\begin{split}
		(A-\lambda_1 I)\vec{v}=(-y,-x-y-z,-y)=(0,0,0)
	\end{split}
\end{equation}
Ahora resolvemos para cada parámetro. Como dos ecuaciones son iguales, nos quedamos con:
\begin{equation}
	\begin{split}
		x+y+z &=0 \\ y &= 0
	\end{split}
\end{equation}
Si definimos \(z = \alpha\), vemos que \(x = -\alpha\), lo que define el siguiente subespacio
vectorial:
\begin{equation}
	\begin{split}
		H_1=\{ (-\alpha,0,\alpha) / \alpha \in \mathbb{R} \}
	\end{split}
\end{equation}
De donde podemos obtener el primer autovector:
\begin{equation}
	\begin{split}
		\vec{v}_1=H_1(1)=(-1,0,1)
	\end{split}
\end{equation}
Repetimos el proceso para el segundo autovalor, obteniendo:
\begin{equation}
	\begin{split}
		(A-\lambda_2 I)\vec{v}=(2x-y,-x+y-z,-y+2z)=(0,0,0)
	\end{split}
\end{equation}
\begin{equation}
	\begin{split}
		2x-y &= 0\\ -x+y-z &= 0\\ -y+2z &= 0
	\end{split}
\end{equation}
Como se cumple que \(-(eq_1 +eq_3)=2 eq_2\), descartamos la egunda ecuación, definimos \(y=\alpha\)
y obtenemos:
\begin{equation}
	\begin{split}
		H_2= \{ (\alpha, 2\alpha, \alpha) / \alpha \in \mathbb{R} \}
	\end{split}
\end{equation}
Y obtenemos el segundo autovector:
\begin{equation}
	\begin{split}
		\vec{v}_2=H_2(1)=(1,2,1)
	\end{split}
\end{equation}
Finalmente repetimos para el tercer autovector:
\begin{equation}
	\begin{split}
		(A-\lambda_3 I)\vec{v}=(-x-y,-x-2y-z,-y-z)=(0,0,0)
	\end{split}
\end{equation}
La segunda ecuacion es la suma de las otras, por la que la descartamos y obtenemos:
\begin{equation}
	\begin{split}
		H_3=\{ (-\alpha, \alpha, -\alpha) / \alpha \in \mathbb{R} \}
	\end{split}
\end{equation}
Para conseguir:
\begin{equation}
	\begin{split}
		\vec{v}_3=(-1,1,-1)
	\end{split}
\end{equation}
Por tanto, nuestra base es:
\begin{equation}
	\begin{split}
		B_D= \{ (-1,0,1),(1,2,1),(-1,1,-1) \}
	\end{split}
\end{equation}
Y la matriz asociada a esta base es:
\begin{equation}
	\begin{split}
		A_{B_D}=\begin{pmatrix}
3 & 0 & 0 \\
0 & 1 & 0 \\
0 & 0 & 4 \\
\end{pmatrix}
	\end{split}
\end{equation}
\item Teniendo \(\vec{v}=(1,2,0)\), queremos \([\vec{v}]_{B_D}\), por lo que expresamos la base
	como variedad lineal e igualamos cada componente:
	\begin{equation}
		\begin{split}
			<B_{D}> = (-x+y-z,2y+z,x+y-z)
		\end{split}
	\end{equation}
	Generando el siguiente sistema:
	\begin{equation}
		\begin{split}
			-x+y-z &= 1 \\ 2y+z &= 2 \\ x+y-z &= 0
		\end{split}
	\end{equation}
	Obteniendo:
	\begin{equation}
		\begin{split}
			[\vec{v}]_{B_D}=(-\frac{1}{2}, \frac{5}{6}, \frac{1}{3})
		\end{split}
	\end{equation}
\item Para obtener \([f(\vec{v})]_{B_D}\), multiplicamos el vector por la matriz de autovalores.
	\begin{equation}
		\begin{split}
			[f(\vec{v})]_{B_D}=A_D \vec{v}=(3 \cdot 1, 1 \cdot 2, 4\cdot 0)=(3,2,0)	
		\end{split}
	\end{equation}
\end{enumerate}
\end{document}
