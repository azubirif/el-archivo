\documentclass{article}
\author{Alejandro Zubiri}
\date{Fri Nov 01 2024}
\title{Entregable 1}

\usepackage{amsmath, amsfonts}

\begin{document}
\maketitle
\tableofcontents
\pagebreak
\section{Ejercicio 1}
Sea $G = {e, a, b, c}$ un conjunto con la operación $\star$ definida por la siguiente tabla, donde $e$ representa
el elemento identidad:
\begin{table}[h!]
    \centering
    \begin{tabular}{l|llll}
    $\star$ & $e$ & $a$ & $b$ & $c$ \\ \hline
    $e$ & $e$ & $a$ & $b$ &  $c$\\
    $a$ & $a$ & $e$ & $c$ & $b$ \\
    $b$ & $b$ & $c$ & $e$ & $a$ \\
    $c$ & $c$ & $b$ & $a$ & $e$
    \end{tabular}
    \end{table}
\subsection{Determina si $(G,\star)$ es un grupo}
Vamos a desarrollar las condiciones que debe cumplir cualquier grupo:
\begin{enumerate}
    \item $\star$ es asociativa\\
    Esto se puede comprobar realizando dos operaciones:
    \begin{align}
        (a\star b)\star c = c\star c = e
    \end{align}
    \begin{align}
    a \star (b\star c) = a\star a = e
    \end{align}
    Como ambos resultados son iguales, podemos confirmar que es asociativa.
    \item Existe el elemento neutro respecto a $\star$\\
    Esto se puede comprobar observando la tabla, donde el elemento neutro es $e$:
    \begin{align}
        a\star e &= a\\ b \star e &= b \\ c \star e &= c
    \end{align}
    \item Existe el inverso para cada elemento\\
    Gracias a la tabla, se observa rápidamente que el inverso de cada elemento es el propio elemento:
    \begin{align}
    a\star a &= e\\ b\star b &= e \\c\star c &= e
    \end{align}
\end{enumerate}
\subsection{$(G,\star)$ es abeliano}
Esto es algo fácilmente observable, viendo que la tabla es simétrica respecto a la diagonal:
\begin{align}
    &a\star b = b\star a = c\\
    &a \star c = c\star a = b\\
    &c \star b = b \star c = a
\end{align}
\pagebreak
\subsection{Determina si $H=\{ e,a \}$ es un subgrupo de $G$.}
Estos elementos forman la tabla
\begin{table}[h!]
    \centering
    \begin{tabular}{l|ll}
    $\star$ & e & a \\ \hline
    e                    & e & a \\
    a                    & a & e
    \end{tabular}
\end{table}\\
De donde podemos comprobar que es
\begin{itemize}
    \item Asociativo: $(a\star e) \star a = a\star a = e$
    \item Existe el neutro, siendo $e$
    \item Existe el invero: $a\star a = e\star e = e$
\end{itemize}
\pagebreak
\section{Ejercicio 2}
Si cierto vector $\vec{u}$ se puede expresar como combinación lineal de los vectores $\vec{v}_{1}$,
$\vec{v}_{2}$, $\vec{v}_{3}$ y, éstos satisfacen:
\begin{itemize}
    \item $\vec{v}_{1} = \vec{w}_{1}+4 \vec{w}_{2}- \vec{w}_{3}$
    \item $\vec{v}_{2} = \vec{w}_{2} - \vec{w}_{3}$
    \item $\vec{v}_{3} = -\vec{w}_{1} - \vec{w}_{2} + \vec{w}_{3}$
\end{itemize}
\subsection{Expresa $\vec{u}$ como CL de $\vec{w}_{1}$, $\vec{w}_{2}$, y $\vec{w}_{3}$}
\begin{equation}
    \begin{split}
        \bar{u} &= a \vec{v}_{1} + b \vec{v}_{2} + c \vec{v}_{3}\\ &= a(\vec{w}_{1}+4 \vec{w}_{2}- \vec{w}_{3})
        + b(\vec{w}_{2} - \vec{w}_{3}) + c(-\vec{w}_{1} - \vec{w}_{2} + \vec{w}_{3})\\
        &= \vec{w}_{1}(a-c) +\vec{w}_{2}(4a+b-c) +\vec{w}_{3} (-a-b+c)
    \end{split}
\end{equation}
\subsection{Si $\vec{v}_{1}$, $\vec{v}_{2}$, $\vec{v}_{3}$ son LI, ¿qué podemos decir de $\vec{w}_{1}$, $\vec{w}_{2}$, y $\vec{w}_{3}$?}
Si $\vec{v}_{1}$, $\vec{v}_{2}$, $\vec{v}_{3}$ son LI, significa que la única solución al sistema
\begin{equation}
    \begin{split}
        a \vec{v}_{1} + b \vec{v}_{2} + c \vec{v}_{3} = \vec{0}
    \end{split}
\end{equation}
Es $a=b=c=0$.\\
Lo que implica que el sistema
\begin{equation}
    \begin{split}
        S = \{ \vec{v}_{1}, \vec{v}_{2}, \vec{v}_{3} \} 
        = \{ \vec{w}_{1}+4 \vec{w}_{2}- \vec{w}_{3}, \vec{w}_{2} - \vec{w}_{3}, -\vec{w}_{1} - \vec{w}_{2} + \vec{w}_{3} \}
    \end{split}
\end{equation}
Es un sistema libre, lo que implica que
\begin{equation}
    \begin{split}
        <S> = \vec{0} = \vec{w}_{1}(a-c) +\vec{w}_{2}(4a+b-c) +\vec{w}_{3} (-a-b+c) \implies a=b=c=0
    \end{split}
\end{equation}
La variedad lineal generada por $\vec{w}_{1}$, $\vec{w}_{2}$, y $\vec{w}_{3}$ es
\begin{equation}
    \begin{split}
        a \vec{w}_{1} + b \vec{w}_{2} + c \vec{w}_{3}
    \end{split}
\end{equation}
Si son LI, entonces la única solución a
\begin{equation}
    \begin{split}
        a \vec{w}_{1} + b \vec{w}_{2} + c \vec{w}_{3} = \vec{0}
    \end{split}
\end{equation}
es $a=b=c=0$. Podemos igualarlo a la variedad lineal generada por $\vec{v}_{1}$, $\vec{v}_{2}$, $\vec{v}_{3}$ ya que
si son LI, tendrán las mismas soluciones al igualarlo a $\vec{0}$.
\begin{equation}
    \begin{split}
        a \vec{w}_{1} + b \vec{w}_{2} + c \vec{w}_{3} = \vec{w}_{1}(a-c) +\vec{w}_{2}(4a+b-c) +\vec{w}_{3} (-a-b+c)
    \end{split}
\end{equation}
Esto genera el sistema
\begin{equation}
    \begin{split}
        a&=a-c\\
        b &= 4a+b-c\\
        c &= -a-b+c
    \end{split}
\end{equation}
Solucionando el sistema podemos obtener que $a=b=c=0$, que implica que los vectores son LI.
\pagebreak
\section{Ejercicio 3}
Determina los valores de $k$ para los cuales el conjunto de vectores $S = \{1 + x, x + x^{2} , 2 - x + kx^{2}\}$
forma un sistema libre o ligado en los siguientes casos:\\
$S$ no es base en ninguno de los casos, ya que tiene $3$ elementos, y para $\mathbb{R}_{3}[x]$ debe tener $4$, y
para $\mathbb{R}_{4}[x]$ debe tener 5.\\
Tampoco puede ser sistema generador en ninguno de los casos, ya que $x^{3} \not \in S \wedge x^{4} \not \in S$.\\
Para ver si $S$ es libre o ligado, podemos analizar el sistema generado al igualar su variedad lineal a $\vec{0}$:
\begin{equation}
    \begin{split}
        <S> = a(1+x) +b(x+x^{2}) +c(2-x+kx^{2}) = x^{2}(a+b-c) +x(b+kc) +a+2c = 0
    \end{split}
\end{equation}
Esto genera el sistema
\begin{equation}
    \begin{split}
        \left.\begin{matrix}
            a+b-c=0 \\
            b+kc=0 \\
            a+2c=0
            \end{matrix}\right\}A' = \begin{pmatrix}
            1 & 1 & -1 & | & 0 \\
            0 & 1 & k & | & 0 \\
            1 & 0 & 2 & | & 0 \\
            \end{pmatrix}
    \end{split}
\end{equation}
Si calculamos el determinante, obtenemos que:
\begin{equation}
    \begin{split}
        |A| = \begin{vmatrix}
            1 & 1 & -1\\
            0 & 1 & k\\
            1 & 0 & 2
        \end{vmatrix} =2+k -(-1)=3+k
    \end{split}
\end{equation}
Si $|A|=0 \implies 3+k =0 \implies k = -3$. Por tanto
\begin{equation}
    \begin{split}
        k \neq -3 \implies rg(A) = 3 \implies \text{S es libre}\\
        k = 3 \implies rg(A) = 2 \implies \text{S es ligado}
    \end{split}
\end{equation}
Esto se cumplirá tanto para $\mathbb{R}_{3}[x]$ como para $\mathbb{R}_{4}[x]$.
\end{document}