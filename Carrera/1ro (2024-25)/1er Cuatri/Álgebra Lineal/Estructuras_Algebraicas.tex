\documentclass{article}
\author{Alejandro Zubiri}
\date{Mon Oct 14 2024}
\title{Estructuras Álgebraicas}

\usepackage{amsmath, amsthm, amsfonts}

\newtheorem{caracteristica}{Definición}
\newtheorem{caracteristica primo}{Teorema}

\begin{document}
\maketitle
\section{Grupos}
Se define un grupo siguiente la siguiente notación
\begin{equation}
    \begin{split}
        (\mathbb{K}, +)
    \end{split}
\end{equation}
Donde $\mathbb{K}$ es un conjunto y $+$ es una operación entre elementos de dicho conjunto (no 
necesáriamente la suma). Estos elementos forman un grupo sí:
\begin{itemize}
    \item La operación es asociativa: $(a+b)+c=a+(b+c)$.
    \item Existe el elemento neutro con respecto a dicha operación en el conjunto:
    $\exists e \in \mathbb{K} : x+e=x \forall x \in \mathbb{K}$.
    \item Existe el inverso para todo elemento: $\forall x \in \mathbb{K} \exists x^{-1} : x+x^{-1} = e$ 
\end{itemize}
Además, este grupo puede ser \textbf{abeliano} si la operación es \textbf{conmutativa}: $a+b=b+a$.
\section{Anillos}
Un anillo está formado por un conjunto $\mathbb{K}$ y dos operaciones $+, \cdot $. Las condiciones
para ser un anillo son:
\begin{itemize}
    \item $(\mathbb{K}, +)$ forman un grupo abeliano.
    \item $ \cdot $ es una operación asociativa.
    \item $ \cdot $ es distributiva con respecto a $+$.
\end{itemize}
Además, este anillo puede ser:
\begin{itemize}
    \item Conmutativo si $ \cdot $ es conmutativa.
    \item Unitario si existe el elemento unitario respecto a $ \cdot $: $\exists u \in \mathbb{K} :
    u \cdot x= x \cdot u = x\forall x \in \mathbb{K}$. 
\end{itemize}
\section{Cuerpos}
Un anillo $(\mathbb{K}; +, \cdot )$ será un cuerpo si:
\begin{itemize}
    \item Es un anillo unitario.
    \item Existe un único elemento inverso respecto a $ \cdot $ para cada elemento: $\forall x \in
    \mathbb{K} \exists \! x^{-1} \in \mathbb{K} : x \cdot  x^{-1} = u$ 
\end{itemize}
Sin embargo, podemos resumir estas condiciones en:
\begin{itemize}
    \item $(\mathbb{K}; + )$ es un grupo abeliano.
    \item $(\mathbb{K} \setminus \{ 0 \}; \cdot )$ es un grupo abeliano (sin el $0$ porque no tiene inverso).
    \item $ \cdot $ es distributivo respecto a $+$. 
\end{itemize}
\section{Cuerpos de módulo $n$}
Estos cuerpos se caracterizan por la siguiente notación:
\begin{equation}
    \begin{split}
        \mathbb{K}_n
    \end{split}
\end{equation}
Donde $n$ es el número de elementos del cuerpo. Hablando vulgarmente, estos cuerpos se comportan como
un reloj respecto a sus operaciones. Por ejemplo, para $\mathbb{K}_2$, teniendo los elementos 
$\{0, 1 \}$, la suma de $1+1 \neq 2$, sino que $1+1=0$. Sería el equivalente a "dar la vuelta".
\begin{caracteristica}
    Si $\exists n \in \mathbb{Z} : n\cdot 1=0$, se dice que tiene característica, $m$, siendo
    $m$ el menor entero $/ m\cdot 1=0$.
\end{caracteristica}
\begin{caracteristica primo}
    Si el subíndice $n$ no es primo, entonces no es un cuerpo.
\end{caracteristica primo}
\begin{proof}[Demostración]
    Supongamos que $n=p\cdot q / p,q \neq 1 \wedge p,q<n$. Entonces
    \begin{equation}
        \begin{split}
            n\cdot 1=(p\cdot q)\cdot 1=(p\cdot 1)(q\cdot 1)=0 \implies p\cdot 1=0 \vee q\cdot 1=0
        \end{split}
    \end{equation}
    Lo que implica que la característica es $p$ o $q$. \#\\
    Por tanto $n$ debe ser primo.\\
    QED
\end{proof}
\end{document}
