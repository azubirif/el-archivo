\documentclass{article}
\author{Alejandro Zubiri}
\title{Formas Cuadráticas}

\usepackage{amsmath, amsfonts, amsthm}

\newtheorem{defi}{Definición}

\begin{document}
\maketitle
Una forma cuadrática es una función $Q: \mathbb{R}^{n} \to \mathbb{R} /
Q = \sum_{i} \sum_{j}a_{ij} \in \mathbb{R}$ con términos de grado $2$. Un ejemplo es:
\begin{equation}
	\begin{split}
		Q: \mathbb{R}^{2} \to \mathbb{R} / Q(x,y)=x^{2}-y^{2}+3xy
	\end{split}
\end{equation}
\section{Matriz asociada}
Sea $Q$ una forma cuadrática y $B_{c}$ la base canónica de $\mathbb{R}^{n}$. Existe
una matriz simétrica de orden $n$ tal que:
\begin{equation}
	\begin{split}
		Q(\vec{v}) = \vec{v}^{T}A \vec{v}
	\end{split}
\end{equation}
Donde $\vec{v}$ son las coordenadas de un vector de $\mathbb{B}_{c}$. La matriz
$A$ es la matriz asociada.\\
Para calcular esta matriz, a cada variable le corresponde una fila y columna
(a $x$ le corresponde la primera fila y colunma, a $y$ la segunda fila y colunma, etc).
\begin{itemize}
	\item Los elementos de la diagonal corresponden a los coeficientes de variables
		al cuadrado $(x_{ii}^{2}$.
	\item El resto de coeficientes son donde cortan cada variable, y son la mitad
		del coeficiente que los acompaña.
\end{itemize}
\section{Signo}
Una forma cuadrática será:
\begin{itemize}
	\item Nula si $Q=0 \forall \vec{v}$.
	\item Definida positiva si $Q > 0 \forall \vec{v}$.
	\item Semidefinida positiva si $Q \geq 0 \forall \vec{v}$.
	\item Definida negativa si $Q < 0 \forall \vec{v}$.
	\item Semidefinida negativa si $Q \leq 0 \forall \vec{v}$.
	\item Indefinida si $Q$ alcanza todo tipo de valores. 
\end{itemize}
\subsection{Cálculo del signo}
\subsubsection{Por autovalores}
Sacando los autovalores $\lambda$ de la matriz asociada:
\begin{itemize}
	\item $Q$ es DP si $\lambda_{i} > 0 \forall i$
	\item $Q$ es SDP si $\lambda_{i} \geq 0 \forall i$
	\item $Q$ es DN si $\lambda_{i} < 0 \forall i$
	\item $Q$ es SDN si $\lambda_{i} \leq 0 \forall i$
	\item $Q$ es IND si $\lambda_{i}$ toma tanto valores
		positivos como negativos.
\end{itemize}
\subsubsection{Por la matriz asociada}
Denotamos por $\Delta_{k}$ el $k-$ésimo menor principal de la
matriz asociada. Un menor principal siempre contiene el primer
elemento.
\begin{itemize}
	\item DP: $\Delta_{k} > 0 \forall k$
	\item DN: $\Delta_{k} < 0 \forall k$
	\item Si $\Delta_{n} \neq 0$ y no estamos en los
		casos anteriores, $Q$ es IND.
	\item $\Delta_{n}=0$ y $\Delta_{k}>0$, es SDP.
	\item $\Delta_{n}=0$ y $(-1)^{k}\Delta_{k} >0$, es SDN.  
\end{itemize}
\section{Cambio de base}
Sea $A'$ la matriz respecto a $\mathbb{B}'$.
\begin{equation}
	\begin{split}
		A'=(\mathbb{B}')^{T}A \mathbb{B}'
	\end{split}
\end{equation}
Y, sea $x'=[u]_{B'}$
\begin{equation}
	\begin{split}
		Q(u)=(x')^{T}A'X'
	\end{split}
\end{equation}
\end{document}
