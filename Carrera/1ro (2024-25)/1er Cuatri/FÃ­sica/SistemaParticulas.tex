\documentclass{article}
\author{Alejandro Zubiri}
\date{Tue Nov 26 2024}
\title{Sistema de partículas}

\usepackage{amsmath, amsfonts, physics, amsthm, tikz, empheq}

\newcommand{\boxedeq}[2]{\begin{empheq}[box={\fboxsep=6pt\fbox}]{align}\label{#1}#2\end{empheq}}

\newtheorem{teorema}{Teorema}

\begin{document}
\maketitle
\tableofcontents
\pagebreak
\section{Centro de masas}
Definimos el centro de masas de un sistema discreto de partículas como
\begin{equation}
    \begin{split}
        \vec{r}_{CM} = \frac{\sum m_{i} \vec{r}_{i}}{\sum m_{i}}
    \end{split}
\end{equation}
Para una distribución continua, tenemos
\begin{equation}
    \begin{split}
        \vec{r}_{CM} = \frac{ \int \vec{r} \dd{m}}{ \int  \dd{m}}
        = \frac{ \int \rho \vec{r} \dd{V}}{ \int \rho  \dd{V}}
    \end{split}
\end{equation}
donde $\rho $ es la densidad en cada posición.
\section{Movimiento del centro de masas}
Si obtenemos la tasa de cambio de la posición respecto al tiempo:
\begin{equation}
    \begin{split}
        \dv{\vec{r}_{CM}}{t} = \dv{t} (\frac{\sum m_{i} \vec{r}_{i}}{\sum m_{i}})
        = \frac{\sum m_{i} \dv{t}(\vec{r}_{i})}{\sum m_{i}}
        = \frac{\sum m_{i} \vec{v}_{i}}{\sum m_{i}} = \vec{V}_{CM}
    \end{split}
\end{equation}
\section{Dinámica del centro de masas}
Definimos el momento total como:
\begin{equation}
    \begin{split}
        \vec{p}_{CM} = \sum m_{i} \vec{v}_{i}
    \end{split}
\end{equation}
Por tanto, si observamos como cambia respecto al tiempo:
\begin{equation}
    \begin{split}
        \dv{t} (\vec{p}_{CM}) = \dv{t}(\sum m_{i} \vec{v}_{i}) = \sum m_{i} \vec{a}_i
        = \sum \vec{F}_{\text{EXT}}
    \end{split}
\end{equation}
Es decir
\begin{equation}
    \begin{split}
        \dv{t} (\vec{p}_{CM}) = \sum \vec{F}_{\text{EXT}} = M_{T} \vec{a}_{CM}
    \end{split}
\end{equation}
\section{Momento angular}
Definimos el momento angular \textbf{total} como:
\begin{equation}
    \begin{split}
        \vec{L}_T = \sum \vec{L}_i = \sum m_{i} \vec{r}_i \times \vec{v}_i
    \end{split}
\end{equation}
Y ahora
\begin{equation}
    \begin{split}
        \dv{L_{T}}{t} = \sum \vec{r}_i \times \vec{F}_i = \sum \vec{r}_i \times
        (\vec{F}_E + \vec{F}_I)
    \end{split}
\end{equation}
Hacemos una distinción entre fuerzas externas $\vec{F}_E$ e internas $\vec{F}_I$:
\begin{equation}
    \begin{split}
        = \sum \vec{r}_i \times \vec{F}_E + \vec{r}_i \times \vec{F}_i = 
        \vec{\tau }_t + \sum _{i,j} (\vec{r}_i - \vec{r}_j) \times \vec{F}_{ij}
    \end{split}
\end{equation}
Por lo general, si las fuerzas son radiales, el último elemento es $0$:
\begin{equation}
    \begin{split}
        \dv{L_{t}}{t} = \sum  \vec{\tau }
    \end{split}
\end{equation}
\section{Energías}
Supongamos que tenemos dos partículas siendo interactuadas entre sí, generando las siguientes
ecuaciones:
\begin{equation}
    \begin{split}
        m_{1} a_{1} &= F_{1} + F_{1,2}\\
        m_{2} a_{2} &= F_{2} + F_{2,1}
    \end{split}
\end{equation}
Podemos realizar la sustitución $F_{1,2} = -F_{2,1}$, multiplicar todo por $ \dd{r}$, y sumarlas:
\begin{equation}
    \begin{split}
        m_{1} a_{1} \dd{r}_{1} +m_{2} a_{2} \dd{r}_{2} &= F_{1} \dd{r}_{1}+ F_{2} \dd{r}_{2}
        + F_{1,2}( \dd{r}_{1} - \dd{r}_2)\\
    \end{split}
\end{equation}
Vamos a desarrollar $\vec{a} \dd{\vec{r}}$:
\begin{equation}
    \begin{split}
        \vec{a} \dd{\vec{r}} = \dd{\vec{r}} \dv{v}{t} = v \dd{v}
    \end{split}
\end{equation}
Obteniendo así:
\begin{equation}
    \begin{split}
        m_{1} v_{1} \dd{v_{1}} +m_{2} v_{2} + \dd{v_{2}} = F_{1} \dd{r}_{1}+ F_{2} \dd{r}_{2}
        + F_{1,2}( \dd{r}_{1} - \dd{r}_2)
    \end{split}
\end{equation}
Podemos integrar a ambos lados para obtener:
\begin{equation}
    \begin{split}
        \Delta T_{1} + \Delta T_{2} = W_{1} + W_{2} + W_{\text{EXT}} 
    \end{split}
\end{equation}
Es decir, que podemos generalizar este principio para afirmar que:
\begin{equation}
    \begin{split}
        \Delta T = W_{\text{INT}} + W_{\text{EXT}} 
    \end{split}
\end{equation}
Si las fuerzas internas son conservativas, obtenemos que:
\begin{equation}
    \begin{split}
        \Delta E = W_{\text{EXT}}
    \end{split}
\end{equation}
\section{Colisiones}
Hay diferentes tipos de colisiones:
\begin{itemize}
    \item Elásticas: se conserva la energía cinética tras la colisión.
    \item Inelástica: no se conserva.
\end{itemize}
Además, definimos la cantidad $Q = \Delta T$, y observamos que:
\begin{itemize}
    \item $Q = 0$: se conserva la energía cinética.
    \item $Q > 0$: se ha ganado energía, colisión endoenergética.
    \item $Q < 0$: se ha perdido energía, colisión exoenergética.
\end{itemize}
\section{Sólido rígido}
Un sistema de partículas continuas con posiciones relativas constantes. En este caso,
tenemos una distribución continua de masa:
\begin{equation}
    \begin{split}
        \vec{r}_{CM} =\frac{ \int \vec{r} \dd{m}}{ \int  \dd{m}}
        = \frac{ \int \rho \vec{r} \dd{V}}{ \int \rho  \dd{V}}
    \end{split}
\end{equation}
\subsection{Rotación de un solído rígido}
En este caso, todas las partículas tienen la misma velocidad angular $\omega $ a lo largo
del mismo eje:
\begin{equation}
    \begin{split}
        \vec{v_{i}} = \vec{\omega }\times \vec{r_{i}} = |\omega| |r_{i}| \sin \theta 
    \end{split}
\end{equation}
Un diagrama rápido nos permite ver lo siguiente:
\begin{center}
    

\tikzset{every picture/.style={line width=0.75pt}} %set default line width to 0.75pt        

\begin{tikzpicture}[x=0.75pt,y=0.75pt,yscale=-1,xscale=1]
%uncomment if require: \path (0,300); %set diagram left start at 0, and has height of 300

%Straight Lines [id:da023861377631871017] 
\draw    (285.87,113) -- (368.53,113) ;
%Straight Lines [id:da5768912366645089] 
\draw    (285.87,113) -- (285.87,253.47) ;
%Straight Lines [id:da9362620452989547] 
\draw    (368.53,113) -- (285.87,253.47) ;
%Shape: Arc [id:dp6275531548125002] 
\draw  [draw opacity=0] (285.87,223.47) .. controls (290.9,223.47) and (295.65,224.71) .. (299.81,226.9) -- (285.87,253.47) -- cycle ; \draw   (285.87,223.47) .. controls (290.9,223.47) and (295.65,224.71) .. (299.81,226.9) ;  
%Shape: Circle [id:dp3409622309110498] 
\draw  [fill={rgb, 255:red, 0; green, 0; blue, 0 }  ,fill opacity=1 ] (364.73,113) .. controls (364.73,110.9) and (366.43,109.2) .. (368.53,109.2) .. controls (370.63,109.2) and (372.33,110.9) .. (372.33,113) .. controls (372.33,115.1) and (370.63,116.8) .. (368.53,116.8) .. controls (366.43,116.8) and (364.73,115.1) .. (364.73,113) -- cycle ;

% Text Node
\draw (380,100.07) node [anchor=north west][inner sep=0.75pt]    {$m_{i}$};
% Text Node
\draw (344,177.07) node [anchor=north west][inner sep=0.75pt]    {$r_{i}$};
% Text Node
\draw (316,82.4) node [anchor=north west][inner sep=0.75pt]    {$R_{i}$};
% Text Node
\draw (290.67,198.73) node [anchor=north west][inner sep=0.75pt]    {$\theta $};


\end{tikzpicture}
\end{center}
\begin{equation}
    \begin{split}
        \sin \theta = \frac{R_{i}}{r_{i}}
    \end{split}
\end{equation}
Lo que implica que:
\begin{equation}
    \begin{split}
        \vec{v_{i}}= |\omega| |r_{i}| \sin \theta = |\omega| |r_{i}|\frac{R_{i}}{r_{i}} = |\omega | |R_{i}|
    \end{split}
\end{equation}
Si multiplicamos por la masa de cada partícula, obtenemos que el momento angular $L_{i}$ de
cada partícula es:
\begin{equation}
    \begin{split}
        L_{i} = m_{i} \omega R_{i}^{2}
    \end{split}
\end{equation}
Para calcular el momento total, sumamos todos los momentos:
\begin{equation}
    \begin{split}
        L_{T}= \sum L_{i} = \sum m_{i} \omega R_{i}^{2} = \omega \sum m_{i} R_{i}^{2}
    \end{split}
\end{equation}
Este último término es el \textbf{momento de inercia}:
\begin{equation}
    \begin{split}
        I = \sum m_{i}R_{i}^{2}
    \end{split}
\end{equation}
Que en distribuciones continuas es:
\begin{equation}
    \begin{split}
        I = \int r^2 \dd{m}= \int r^{2} \rho  \dd{V}
    \end{split}
\end{equation}
Por tanto, el momento angular del sólido es:
\begin{equation}
    \begin{split}
        L = \omega I
    \end{split}
\end{equation}
\begin{teorema}[Teorema de Steiner o de los ejes paralelos]
    Dados dos ejes paralelos y un sólido en rotación, la relación entre el momento de inercia
    alrededor del centro de masa $I_{c}$ y alrededor de un eje paralelo $I'$ a una distancia $d$
    es:
    \begin{equation}
        \begin{split}
            I' = I_{c} + M_{t}d^{2}
        \end{split}
    \end{equation}
\end{teorema}
\begin{proof}
    Tenemos el siguiente diagrama de un objeto rotando alrededor de los dos ejes:
    \begin{center}
        

\tikzset{every picture/.style={line width=0.75pt}} %set default line width to 0.75pt        

\begin{tikzpicture}[x=0.75pt,y=0.75pt,yscale=-1,xscale=1]
%uncomment if require: \path (0,300); %set diagram left start at 0, and has height of 300

%Straight Lines [id:da6531898193555183] 
\draw    (167,164.6) -- (54,277.6) ;
%Straight Lines [id:da6246528418972856] 
\draw    (540,164.6) -- (167,164.6) ;
%Straight Lines [id:da6099367582335531] 
\draw    (167,19.6) -- (167,164.6) ;
%Straight Lines [id:da4820162860689021] 
\draw [color={rgb, 255:red, 74; green, 144; blue, 226 }  ,draw opacity=1 ]   (385,165) -- (272,278) ;
%Straight Lines [id:da1605885229137145] 
\draw [color={rgb, 255:red, 74; green, 144; blue, 226 }  ,draw opacity=1 ]   (598,165) -- (385,165) ;
%Straight Lines [id:da3659213855063408] 
\draw [color={rgb, 255:red, 74; green, 144; blue, 226 }  ,draw opacity=1 ]   (385,20) -- (385,165) ;
%Straight Lines [id:da15119634463509013] 
\draw  [dash pattern={on 4.5pt off 4.5pt}]  (167,106.6) -- (386,106.6) ;
%Shape: Circle [id:dp5309484932224788] 
\draw  [fill={rgb, 255:red, 0; green, 0; blue, 0 }  ,fill opacity=1 ] (400.4,238) .. controls (400.4,234.35) and (403.35,231.4) .. (407,231.4) .. controls (410.65,231.4) and (413.6,234.35) .. (413.6,238) .. controls (413.6,241.65) and (410.65,244.6) .. (407,244.6) .. controls (403.35,244.6) and (400.4,241.65) .. (400.4,238) -- cycle ;
%Straight Lines [id:da522053039709147] 
\draw  [dash pattern={on 4.5pt off 4.5pt}]  (385,165) -- (407,238) ;
%Straight Lines [id:da788440458694861] 
\draw  [dash pattern={on 4.5pt off 4.5pt}]  (167,164.6) -- (407,238) ;

% Text Node
\draw (145,76.4) node [anchor=north west][inner sep=0.75pt]    {$I_{c}$};
% Text Node
\draw (365,69.4) node [anchor=north west][inner sep=0.75pt]    {$I'$};
% Text Node
\draw (276,86.4) node [anchor=north west][inner sep=0.75pt]    {$d$};
% Text Node
\draw (412,240.4) node [anchor=north west][inner sep=0.75pt]    {$m_{i}$};
% Text Node
\draw (403,183.4) node [anchor=north west][inner sep=0.75pt]    {$R'_{i}$};
% Text Node
\draw (203,184.4) node [anchor=north west][inner sep=0.75pt]    {$R_{i}$};


\end{tikzpicture}
    \end{center}
    El momento de inercia a lo largo del eje $I'$ es:
    \begin{equation}
        \begin{split}
            I' = \sum m_{i} r_{i}^{2}
        \end{split}
    \end{equation}
    Observando el diagrama, podemos ver que:
    \begin{equation}
        \begin{split}
            R_{i}^{2} = x^{2}+y^{2}=x_{I'}^{2}+y_{I'}^{2} 
            = x^{2}_{I}+y_{I}^{2}+2y_{i}d +d^{2} = R_{I}^{2}+2y_{i}d +d^{2}
        \end{split}
    \end{equation}
    Si introducimos esto en nuestra ecuación del momento de intercia paralelo obtenemos:
    \begin{equation}
        \begin{split}
            I' = \sum m_{i}(R_{I}^{2}+2y_{i}d +d^{2})
        \end{split}
    \end{equation}
    El primer término $\sum m_{i} R_{i}^{2} = I_{C}$, el segundo $d^{2} \sum m_{i} = M_{t}d^{2}$.
    Del tercer término $2d \sum m_{i} y_{Ic}$, es fácilmente reconocible que:
    \begin{equation}
        \begin{split}
            Y_{CM} = \frac{\sum m_{i} y_{ic}}{\sum m_{i}} \implies \sum m_{i} y_{ic} = Y_{CM,c} M_{T}
        \end{split}
    \end{equation}
    Como estamos midiendo la distancia al centro de masas desde el centro de masas:
    \begin{equation}
        \begin{split}
            2d \sum m_{i} y_{Ic} = 0
        \end{split}
    \end{equation}
    Obteniendo así:
    \begin{equation}
        \begin{split}
            I' = I_{c} + M_{t} d^{2}
        \end{split}
    \end{equation}
\end{proof}
\section{Segunda ley de Newton para rotación}
Partimos de que
\begin{equation}
    \begin{split}
        \dv{L}{t} = \vec{\tau }
    \end{split}
\end{equation}
A su vez, $\vec{L} = \vec{\omega }I$, por tanto
\begin{equation}
    \begin{split}
        \dv{L}{t} = \alpha I + \omega \dot{I}
    \end{split}
\end{equation}
Si la masa del objeto es constante, y la distancia al eje no cambia, $\dot{I} = 0$, por tanto:
\begin{equation}
    \begin{split}
        \boxed{\vec{\tau } = I \vec{\alpha }}
    \end{split}
\end{equation}
\section{Energía cinética}
Partimos de la energía cinética de una única partícula:
\begin{equation}
    \begin{split}
        T_{i} = \frac{1}{2} m_{i}v_{i}^{2} = \frac{1}{2} m_{i}(v_{CM}+v_{rel})^{2} = \frac{1}{2}m_{i}(v_{CM}^{2}+2v_{cm}v_{rel}+v_{rel}^{2})
    \end{split}
\end{equation}
Ahora sumamos para obtener la energía cinética total:
\begin{equation}
    \begin{split}
        T_{T} &= \sum \frac{1}{2} m_{i} v_{CM}^{2} + \sum m_{i}v_{cm}v_{rel} + \sum \frac{1}{2}m_{i}v_{rel}^{2}\\
        &=\frac{1}{2} v_{CM}^{2} \sum m_{i} + \frac{1}{2} \omega^{2} \sum m_{i} r_{i}^{2} + v_{CM} \sum m_{i} v_{rel}\\
        &=\frac{1}{2} v_{CM}^{2} M_{T} + \frac{1}{2} \omega^{2} I + v_{CM} \sum m_{i} v_{rel}
    \end{split}
\end{equation}
Puesto que estamos midiendo la velocidad del centro de masas respecto a sí mismo, $\sum m_{i} v_{rel} = 0$, por tanto:
\begin{equation}
    \begin{split}
        \boxed{T = \frac{1}{2}M v_{CM}^{2} + \frac{1}{2} \omega^{2} I}
    \end{split}
\end{equation}
\end{document}