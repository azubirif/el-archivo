\documentclass{article}
\author{Alejandro Zubiri}
\title{Introducción al diseño experimental}

\usepackage[margin=1.1in]{geometry}
\usepackage{amsmath, amsthm, amsfonts}

\usepackage{fancyhdr}
\pagestyle{fancy}
\fancyfoot[R]{Página \thepage}
\fancyfoot[L]{Alejandro Zubiri}

\begin{document}
\maketitle
\tableofcontents
\pagebreak

\section{Introducción}
Cuando realizamos un experimento varias veces, se suelen producir variaciones, que se conocen como \textbf{ruido}.
Las posibles fuentes de error son:
\begin{itemize}
	\item Error experimental: factores tanto conocidos como desconocidos.
	\item Confusión entre correlación y causalidad.
	\item Complejidad de los elementos estudiados.
\end{itemize}
Cuando realizamos experimentos, queremos los métodos estadísticos que minimicen
este error.
\section{Tipos de variabilidad}
\subsection{Sistemática planificada}
\begin{itemize}
	\item Planificada por el experimentador.
	\item Por las condiciones del experimento.
\end{itemize}
\subsection{Sistemática no planificada}
Debido a causas externas desconocidas que producen resultados sesgados.
\section{Definiciones}
\subsection{Unidad experimental}
Objetos, individuos, espacio o tiempo en los que se experimenta.
\subsection{Variable de interés}
Lo que se desea estudiar.
\subsection{Observación experimental}
Cada medición que realizamos.
\subsection{Tamaño del experimento}
Número total de observaciones recogidas.
\subsection{Factores}
Variables independientes que pueden influir en los resultados.
\subsection{Niveles}
Cada uno de los resultados de un factor.
\subsection{Tratamiento}
Como se combinan los diferentes niveles de un factor.
\section{Básicos del diseño experimental}
\subsection{Aleatorización}
Tiene el objetivo de evitar el sesgo en el experimento. Se asignan al azar los tratamientos.
También se evita la dependencia de las observaciones.
\subsection{Bloqueo}
Se dividen las unidades experimentales en grupos llamados \textbf{bloques}. Las
observaciones de cada bloque deben tener condiciones experimentales similares.
\subsection{Factorización}
Cruzamiento de los niveles de todos los factores de tratamiento en todas las
combinaciones posibles. Permite detectar la existencia de interacciones entre
tratamientos. Además, es mucho más eficiente.
\subsection{Diseño Experimental}
Regla con la que se determina la asignación de las unidades experimentales:
\begin{itemize}
	\item Completamente aleatorizado.
	\item Factor bloque.
	\item Factorial a dos niveles.
\end{itemize}
\section{Diseño completamente aleatorizado}
Los tamaños muestrales no tienen por qué ser iguales, aunque tiene ventajas:
\begin{itemize}
	\item Asegura que cada tratamiento contribuye igual.
	\item Reduce problemas derivados de incumplimiento de hipótesis.
	\item Incrementa la potencia del test.
\end{itemize}
Realizamos la siguiente pregunta: ¿Son todas las medias iguales?\\
Las observaciones del tratamiento $j$  provienen de una población común $\mu$,
un efecto del tratamiento $A_{j}$ y un error aleatorio $e_{ij}$.
\begin{equation}
	\begin{split}
		y_{ij}=\mu + A_{i} + e_{ij}
	\end{split}
\end{equation}
\section{Análisis de la varianza}
Queremos saber si cada factor tiene un efecto significativo sobre $Y$, o si hay
dependencia entre factores.
\subsection{Condiciones}
\begin{itemize}
	\item Independencia: tamaño total de la muestra $< 10\%$.
	\item Normalidad: La variable cuantitativa se distribuye normalmente en cada
		grupo.
	\item Homocedasticidad: varianza constante entre grupos.
\end{itemize}
Tenemos las siguientes tipos de varianza:
\begin{itemize}
	\item Suma de cuadrados total $(SCT)$
	\item Suma de cuadrados del tratamiento $(SCTr)$
	\item Suma de cuadrados residual $(SCR)$ 
\end{itemize}
Se cumple que:
\begin{equation}
	\begin{split}
		SCT = SCTr + SCR
	\end{split}
\end{equation}
\begin{itemize}
	\item SCT: variabilidad total de los datos:
\begin{equation}
	\begin{split}
		SCT = \sum_{i}\sum_{j} (y_{ij}-\bar{y})^{2}
	\end{split}
\end{equation}
	\item SCTr: variabilidad de los datos asociada al efector del factor sobre
		la media, ponderado por el número de observaciones:
\begin{equation}
	\begin{split}
		SCTr = \sum_{i}(\bar{y_{i}}-\bar{y})^{2}n_{i}
	\end{split}
\end{equation}
	\item SCR: variabilidad no debida a la variable factor. Es la suma de los
		cuadrados dentro de los grupos:
\begin{equation}
	\begin{split}
		SCR = \sum_{i}\sum_{j}(y_{ij}-\bar{y}_{i})^{2}=SCT-SCTr
	\end{split}
\end{equation}
\end{itemize}
Ahora, podemos analizar cada grupo
\begin{table}[h]
	\centering
	\begin{tabular}{c | c | c|c}
		Variabilidad & Suma de cuadrados & Grados de libertad & Cuadrados medios\\
		\hline
		Intergrupo & SCTr & $k-1$ & $CMTr = \frac{SCTr}{k-1}$ \\
		Intragrupos & SCR & $n-1$ & $CMR = \frac{SCR}{n-k}$ \\
		Total & SCT & $n-1$ & $CMT = \frac{SCT}{n-1}$  
	\end{tabular}
\end{table}
Además, se cumple que:
\begin{equation}
	\begin{split}
		\frac{intervarianza}{intravarianza} = \frac{s^{2}_{Tr}}{s^{2}_{R}}
		\approx F_{k-1,n-k}
	\end{split}
\end{equation}
Siendo $F$ la $F$ de Fischer. Si se cumple que:
\begin{equation}
	\begin{split}
		\frac{s^{2}_{Tr}}{s^{2}_{r}} > F_{k-1,n-k}
	\end{split}
\end{equation}
\begin{centering}
	\huge Rechazamos la igualdad de medias.	
\end{centering}

\end{document}

