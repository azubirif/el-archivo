\documentclass{article}
\author{Alejandro Zubiri}
\date{Wed Dec 18 2024}
\title{Regresión Múltiple}

\usepackage{geometry, amsmath, amsthm, amsfonts}

 \geometry{
 a4paper,
 total={170mm,257mm},
 left=20mm,
 top=20mm,
 }

\begin{document}
\maketitle
\tableofcontents
\pagebreak
Vamos a crear una recta de la forma:
\begin{equation}
    \begin{split}
        y = \beta _{0} + \beta _{1} x_{1i}+\dots + \beta _{i} x_{ik} + e_{i}
    \end{split}
\end{equation}
Que podemos expresar de forma matricial de la siguiente forma:
\begin{equation}
    \begin{split}
        Y = X\beta + e
    \end{split}
\end{equation}
Obtendremos los coeficientes mediante:
\begin{equation}
    \begin{split}
        \beta = (X^{T}X)^{-1}X^{T}Y
    \end{split}
\end{equation}
Para estimar la varianza $\sigma^{2}$:
\begin{equation}
    \begin{split}
        S^{2}_{e} = \frac{\sum e^{2}}{n}
    \end{split}
\end{equation}
Que es la varianza residual. Para que sea \textbf{sesgado}, usamos:
\begin{equation}
    \begin{split}
        S^{2}_{R} = \frac{\sum e^{2}}{n-p}
    \end{split}
\end{equation}
Donde $p$ es el número de parámetros beta. Se cumple que:
\begin{equation}
    \begin{split}
        E(S^{2}_{R}) &= \sigma^{2}\\
        e &\to N(0,\sigma^{2})\\
        y &\to N(\beta _{0} + \beta _{1} x_{1i}+\dots, \sigma^{2})
    \end{split}
\end{equation}
El coeficiente de determinación corregido es aquel que funcione para muchos valores:
\begin{equation}
    \begin{split}
        R^{2}=1- \frac{S^{2}_{R}}{S^{2}_{y}}
    \end{split}
\end{equation}
\section{Diagnosis del modelo}
Debemos comprobar las hipótesis del modelo:
\begin{itemize}
    \item Linealidad: residuos frente a valores. La distribución debe ser aleatoria.
    \item Homocedasticidad: residuos frente a $x$, que debe ser aleatoria.
    \item Independencia.
    \item Normalidad.
\end{itemize}
\section{Transformaciones}
Son las vistas previamente. Solo se deben transformar las variables sin comportamiento lineal.
\section{Variables binarias}
Son aquellas que toman valores de $0$ o $1$, y representan ausencia o presencia de propiedad.
Deben ser excluyentes.
\section{Contraste de parámetros}
Queremos ver si las variables son significativas, y existen tres métodos:
\begin{itemize}
    \item Si $p<0.05$, $\beta $ es significativa.
    \item $t-$valor: si $|t|>2$, $\beta $ es significativa.
    \item Error estándar: calculamos un intervalo de confianza para $\beta $ con $\alpha =0.05$:
    \begin{equation}
        \begin{split}
            \beta \pm 2 SE(\beta )
        \end{split}
    \end{equation}
    Si el $0$ está incluido, $\beta $ \textbf{NO} es significativa
\end{itemize}
\end{document}