\documentclass{article}
\author{Alejandro Zubiri}
\date{Fri Oct 11 2024}
\title{R Studio}

\begin{document}
\maketitle
\section{Atajos de teclado}
\begin{itemize}
    \item \verb|Ctrl + Alt + K|: ver atajos de teclado.
    \item \verb|Ctrl + Alt + I|: añadir bloque de código.
    \item \verb|Ctrl + Shift + Enter|: ejecutar bloque actual.
\end{itemize}
\section{Gráficos}
\subsection{Histogramas}
\begin{verbatim}
    hist(variable, xlab="Nombre Eje X", ylab="Nombre Eje Y", main="Nombre Gráfico",
    col="color", xaxp=c(lim_inf, lim_sup, n_intervalos), breaks = n_clases)
\end{verbatim}
\textbf{Número de clases}:\\
Podemos elegir el algoritmo a seguir (predeterminado es Sturges)
\begin{itemize}
    \item Sturges (\verb|"sturges"|)
    \item Freedman-diaconis (\verb|"fd"|)
    \item Scott (\verb|"scott"|)
\end{itemize}
\subsection{Diagrama de caja}
\begin{verbatim}
    boxplot(datos,xlab="Nombre Eje X", ylab="Nombre Eje Y", main="Nombre Gráfico",
    col="color", xlim=c(lim_inf, lim_sup), ylim=c(lim_inf, lim_sup))
\end{verbatim}
Si queremos hacer dos diagramas de caja en función de otra variable, declaramos \verb|datos~datos$variable|.
\section{Comandos}
\begin{itemize}
    \item \verb|par(mfrow=c(1,2))|: crear regiones con una fila y dos columnas.
    \item \verb|range(vector)|: devuelve el valor más pequeño y el más grande.
    \item \verb|seq(inicio, final, saltos)|: crea un vector que empieza en \verb|inicio|, acaba en \verb|final|, y salto por \verb|saltos|.
    \item \verb|cut(variable, secuencia, right)|: divide \verb|variable| siguiendo la regla \verb|secuencia|. Si \verb|right|, los
    valores se cortan por la derecha.
    \item \verb|summary(variable)|: proporciona, de la \verb|variable|, su mínimo, primer cuartil, mediana, media, tercer
    cuartil, y máximo.
    \item \verb|descr(variable)| (de la librería \verb|summarytools|): proporciona información avanzada de la variable. Si
    queremos ignorar una determinada variable, le indicamos al programa que lo trate como cualitativo:
    \begin{verbatim}
        datos$variable <- as.factor(datos$variable)
    \end{verbatim}
    \item \verb|quantile(variable, probs=seq(inicio, final, salto))|: da los\\cuantiles de \verb|variable|.
    Podemos pedirle un cuantil en específico con \verb|quantile(variable, .50)| o \verb|quantile(variable, c(0.13, 0.90)|.
\end{itemize}
\section{Librerías}
Para cargar una librería, escribimos \verb|library(nombre_libreria)|.
\begin{itemize}
    \item \verb|summarytools|: información más avanzada.
\end{itemize}

\end{document}
