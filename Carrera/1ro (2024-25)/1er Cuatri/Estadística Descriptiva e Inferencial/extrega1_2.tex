\documentclass{article}
\author{Alejandro Zubiri}
\date{Sun Dec 01 2024}
\title{Extrega 1 - Segunda Entrega}

\usepackage{amsmath, amsthm, amsfonts}

\newtheorem*{definicion}{Definición}

\begin{document}
\maketitle
\begin{definicion}[Precisión]
    Definimos la precisión de un estimador $\hat{\theta }$ como la proximidad entre las medidas
    de una misma muestra. En concreto, se calcula mediante el \textbf{error de muestreo}:
    \begin{equation}
        \begin{split}
            e(\hat{\theta }) = \frac{\sigma }{\sqrt{n}}
        \end{split}
    \end{equation}
    Donde $\sigma $ es la desviación típica del estimador, y $n$ el número de elementos de la
    muestra.
\end{definicion}
\section*{Ejercicios 2, apartado B}
    Queremos saber qué estimador es más preciso.\\
    La varianza de los dos estimadores dados es, respectivamente:
    \begin{equation}
        \begin{split}
            var (\hat{\theta }_1) &= \frac{9}{16} \sigma^{2}\\
            var (\hat{\theta }_2) &= \frac{3}{8}\sigma^{2}
        \end{split}
    \end{equation}
    Podemos calcular el error de muestro de cada estimador mediante la ecuación del error de
    muestreo:
    \begin{equation}
        \begin{split}
            e(\hat{\theta }_1) &= \frac{\sqrt{var (\hat{\theta }_1)}}{\sqrt{n}} =
            \frac{1}{2} \frac{3}{4} \sigma = \frac{3}{8}\sigma = 0.375\sigma  \\
            e(\hat{\theta }_2) &= \frac{\sqrt{var (\hat{\theta }_2)}}{\sqrt{n}}=
            \frac{1}{2} \frac{\sqrt{3}}{\sqrt{8}} \sigma  = \sqrt{\frac{3}{48}}\sigma
            \approx 0.30618\sigma 
        \end{split}
    \end{equation}
    Comparando ambos errores:
    \begin{equation}
        \begin{split}
            0.375\sigma &> 0.30618\sigma \\
            0.375 &> 0.30618
        \end{split}
    \end{equation}
    Esto nos permite llegar a la conclusión de que \textbf{el estimador 1 es más preciso}.
\end{document}