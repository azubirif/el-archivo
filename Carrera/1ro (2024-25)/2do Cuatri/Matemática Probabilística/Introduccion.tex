%! TEX root = Probabilidad.tex

\documentclass{../Probabilidad.tex}

\begin{document}
\section{Experimentos aleatorios}
\subsection{Espacio muestral}
Son sucesos aleatorios repetidos cuyos resultados no se pueden determinar de
antemano. El objetivo es describir el fenónemo desde un punto de vista aleatorio
que describa el proceso.
\begin{defin}
	El espacio muestral es el conjunto formado por todos los posibles resultados:
	\begin{equation}
		\begin{split}
		\Omega = \{ \omega_{1},\dots ,\omega_{n} \}
		\end{split}
	\end{equation}
	Y tenemos diferentes tipos:
	\begin{itemize}
		\item Discreto: finito o numerable.
		\item Continuo.
	\end{itemize}
\end{defin}
\begin{defin}
	Un suceso es cualquier subconjunto del espacio muestral
\end{defin}

\end{document}
