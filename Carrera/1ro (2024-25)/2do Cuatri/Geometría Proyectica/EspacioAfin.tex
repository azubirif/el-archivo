%! TEX root = Geometria.tex

\documentclass{../Geometria.tex}

\begin{document}
\pagebreak
\section{Espacio Afín}
\begin{defin}
	Dados un conjunto de elementos, siendo estos puntos, $A$, y un espacio
	vectorial $\mathbb{V}$, llamamos el espacio afín a la terna $(A,\mathbb{V},
	\varphi)$, siendo $\varphi$ una aplicación entre elementos de $A$, tal que:
	\begin{equation}
		\begin{split}
			\varphi: A \times A \mapsto \mathbb{V}
		\end{split}
	\end{equation}
	Esta terna debe cumplir que:
	\begin{itemize}
		\item $\forall p \in A \wedge \vec{v} \in \mathbb{V}, \exists! Q \in A
			 / \varphi(P,Q) = \vec{PQ}=\vec{u}=Q-P$
		\item Relación de Chasles: $\forall P,Q,R \in A \wedge
			\varphi (P,Q) + \varphi (Q,R) = \varphi (P,R)$
			\begin{equation}
				\begin{split}
					\vec{PQ} + \vec{QR} = \vec{PR}
				\end{split}
			\end{equation}
	\end{itemize}
	\begin{proof}[Demostración]
		\begin{itemize}
			\item $\vec{PQ}=Q-P$
			\item $\vec{QR}=R-Q$ 
		\end{itemize}
		\begin{equation}
			\begin{split}
				\vec{PQ}+\vec{QR} = Q-P +R-Q = R-P = \vec{PR} 
			\end{split}
		\end{equation}
	\end{proof}
	La dimensión del espacio afín va a ser la dimensión de $\mathbb{V}$. 
\end{defin}
\subsection{Propiedades del espacio afín}
\begin{enumerate}
	\item $\forall P \in A, \varphi(P,P)=0$
	\item $\varphi(P,Q)=0 \iff P=Q$
	\item $\forall P,Q \in A, \varphi(P,Q) = -\varphi(Q,P)$
	\item Regla del paralelogramo: $\forall P,Q,R,S \in A$:
		\begin{equation}
			\begin{split}
				\varphi(P,Q)=\varphi(R,S) \iff \varphi (P,R) = \varphi (Q,S)
			\end{split}
		\end{equation}
\end{enumerate}
\end{document}
