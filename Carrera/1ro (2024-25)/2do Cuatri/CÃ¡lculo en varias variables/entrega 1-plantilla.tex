\documentclass[12pt,reqno]{article}


\usepackage[spanish,es-noquoting]{babel}
\usepackage{amsmath}
\usepackage{amssymb}
\usepackage{bm}
\usepackage{enumerate}
\usepackage[shortlabels]{enumitem}
\setlist{topsep=1.5em, itemsep=1.5em}
\usepackage[exercisename=Problema,solutionname=Solución]{exercises}
\usepackage{hyperref}
\usepackage{graphicx}
\usepackage{mathtools}
\usepackage{wrapfig2}

\newcommand{\bbR}{\mathbb{R}}
\newcommand{\cC}{\mathcal{C}}
\newcommand{\cQ}{\mathcal{Q}}
\newcommand{\cS}{\mathcal{S}}
\newcommand{\dparcial}[2]{\frac{\partial#1}{\partial#2}}
\newcommand{\derivada}[2]{\frac{d#1}{d#2}}
\newcommand{\evaluar}[2]{\left.#1\right|_{#2}}
\newcommand{\exterior}{\text{ext }}
\newcommand{\figref}[1]{\text{fig. }\ref{#1}}
\newcommand{\frontera}{\partial}
\newcommand{\grad}{\text{grad }}
\newcommand{\half}{\frac{1}{2}}
\newcommand{\interior}{\text{int }}
\newcommand{\norm}[1]{\left\|#1\right\|}
\newcommand{\sgn}{\text{sgn}}
\newcommand{\tr}{\text{Tr }}
\newcommand{\union}{\cup}

\title{Entrega 1}
\author{Alejandro Zubiri Funes}
\date{\today}

\begin{document}
	\maketitle
	\section*{Problema 1 [3 puntos]}
	Si $A=(a_{ij})$ y $S=(s_{ij})$ son matrices $n\times n$ tales que
	\begin{align*}
		a_{ji}&=-a_{ij}\\
		s_{ji}&=s_{ij}
	\end{align*}
	demuestra que
	\begin{equation*}
		\sum_{i=1}^{n}\sum_{j=1}^{n} a_{ij}\,s_{ij}=0.
	\end{equation*}
	
	\subsection*{Solución}
Si empezamos analizando las consecuencias de ambas restricciones, obtenemos
lo siguiente:
\begin{itemize}
	\item Para la primera matriz, si $a_{ij}=-a_{ji}$, tenemos que la matriz
		es \textbf{antisimétrica}, es decir, los elementos por debajo de la
		diagonal son los opuestos a los de por encima de la diagonal. Y respecto
		a la diagonal, se tiene que $a_{ii}=-a_{ii}$, por lo que la única
		posibilidad es una diagonal llena de ceros.
	\item Las propiedades de la segunda matriz son análogas, excepto que es
		\textbf{simétrica}, es decir, los elementos por debajo de la diagonal son
		los mismos a los de por encima de la diagonal. Con esto, obtenemos
		las siguientes matrices:
\end{itemize}
\[A=
	\begin{pmatrix} 
		0 & a_{12} & \dots & a_{1n}\\
		-a_{12} & 0 & \dots & a_{2n}\\
		\dots &\dots &\dots &\dots \\
		-a_{1n}&-a_{2n} & \dots & 0
	\end{pmatrix} 
\]
\[S=
	\begin{pmatrix} 
		0 & a_{12} & \dots & a_{1n}\\
		a_{12} & 0 & \dots & a_{2n}\\
		\dots &\dots &\dots &\dots \\
		a_{1n}&a_{2n} & \dots & 0
	\end{pmatrix} 
\]
Ahora podemos empezar a desarrollar el sumatorio. Para empezar, todo producto de
la forma $a_{ii}\cdot s_{ij}$ o $a_{ij}\cdot s_{ii}$ será $0$ debido a las
propiedades explicadas anteriormente. Respecto al resto de elementos, sabemos
que para cada producto $a_{ij}s_{ij}$ existe otro igual con signo negativo.
Como los elementos de la diagonal son $0$ por sí solos, el resto de los
productos se irán cancelando uno a uno hasta obtener que el resultado es $0$. 
	\newpage
	
	\section*{Problema 2 [1'5 puntos + 1'5 puntos]}
	En las siguientes expresiones el rango de definición de los índices es $\{1,2,\ldots,n\}$, además se usa el criterio de suma de Einstein. Llega, en cada caso, a una expresión algebraica más sencilla.
	\begin{enumerate}[(a)]
		\item $\delta_{ii}$.
		\subsection*{Solución}
		Al utilizar el criterio de suma de Einstein, sumamos los índices
		repetidos hasta un valor $n$:
		\[
			\sum_{i=1}^{n} \delta_{ii}
		\]
		Como la delta de Kronecker vale $1$ cuando los índices son iguales,
		todos los elementos del sumatorio van a ser $1$, y por tanto:
		\[
			= \sum_{i=1}^{n}1 = n
		\]
		\item $\delta_{ij}\,\delta_{jk}$.
		\subsection*{Solución}
		De nuevo, expresamos esto como un sumatorio:
		\[
			= \sum_{j=1}^{n}\delta_{ij}\delta_{jk}
		\]
		Ahora, para que ambos valores valgan $1$, se tiene que cumplir que
		\[
			i=j, \quad j=k \implies i=k
		\]
		Para que esto ocurra, hay dos posibilidades:
		\begin{itemize}
			\item Si $i = k \wedge i < n$, entonces hay algún valor $j$ 
				donde $j=i$ y por tanto $j = k$, obteniendo así
				$\delta_{jj}\delta_{jj}=1$.
			\item Si $i\neq k$, entonces en ningún punto ambas delta valdrán
				$1$, y por tanto el sumatorio total será $0$.  
		\end{itemize}
	\end{enumerate}
	
	\section*{Problema 3 [2 puntos]}
	Expande el siguiente sumatorio
	\begin{equation*}
		\sum_{i=1}^{3}\sum_{j=1}^{3}\sum_{k=1}^{3}\epsilon_{ijk}\,a_{1i}\,a_{2j}\,a_{3k}
	\end{equation*}
	después da el valor correspondiente a cada $\epsilon$ y comprueba que obtienes el determinante de la matriz $(a_{ij})$.
	\subsection*{Solución}
	\begin{equation*}
		\begin{split}
		&\sum_{i=1}^{3}\sum_{j=1}^{3}\sum_{k=1}^{3}\epsilon_{ijk}\,a_{1i}\,a_{2j}\,a_{3k}\\
		&= \sum_{i=1}^{3}\sum_{j=1}^{3} \varepsilon_{ij1}a_{1 i}a_{2j}a_{31}+\varepsilon_{ij 2}a_{1i}a_{2j}a_{32} + \varepsilon_{ij 3}a_{1i}a_{2j}a_{33}\\
		&= \sum_{i=1}^{3} \varepsilon_{i 11}a_{1i}a_{21}a_{31}+\varepsilon_{i 12}a_{1i}a_{21}a_{32}+\varepsilon_{i 13}a_{1i}a_{21}a_{33}\\
		&+ \varepsilon_{i 21}a_{1i}a_{22}a_{31}+\varepsilon_{i 22}a_{1i}a_{22}a_{32}+\varepsilon_{i 23}a_{1i}a_{22}a_{33}\\
		&+ \varepsilon_{i 31}a_{1i}a_{23}a_{31}+\varepsilon_{i 32}a_{1i}a_{23}a_{32}+\varepsilon_{i 33}a_{1i}a_{23}a_{33}
		\end{split}
	\end{equation*}
	Como todos los símbolos de Levi-Civita con símbolos repetidos son igual a $0$, podemos simplificar este sumatorio y continuar:
	\begin{equation*}
		\begin{split}
			&= \sum_{i=1}^{3}\varepsilon_{i 12}a_{1i}a_{21}a_{32}+\varepsilon_{i 13}a_{1i}a_{21}a_{33}+\varepsilon_{i 21}a_{1i}a_{22}a_{31}\\
			&+ \varepsilon_{i 23}a_{1i}a_{22}a_{33}+\varepsilon_{i 31}a_{1i}a_{23}a_{31}+\varepsilon_{i 32}a_{1i}a_{23}a_{32}
		\end{split}
	\end{equation*}
	Con el objetivo de simplificar, para cada índica $i$, solo vamos a escribir aquellos términos cuyo símbolo de Levi-Civita no sea $0$, es decir, donde
	los índices no se repitan. Por ejemplo, para $i=1$, el primer, segundo y tercer término son $0$,y así sucesivamente. Con esto obtenemos:
	\begin{equation*}
		\begin{split}
			&= \varepsilon_{123}a_{11}a_{22}a_{33}+\varepsilon_{132}a_{11}a_{23}a_{32}\\
			&+ \varepsilon_{213}a_{12}a_{21}a_{33}+\varepsilon_{231}a_{12}a_{23}a_{31}\\
			&+ \varepsilon_{312}a_{13}a_{21}a_{32}+\varepsilon_{321}a_{13}a_{22}a_{31}
		\end{split}
	\end{equation*}
	Ahora podemos pasar a evaluar cada símbolo, obteniendo:
	\begin{equation*}
		\begin{split}
			&\varepsilon_{123}=1\quad \varepsilon_{231}=1\\
			&\varepsilon_{132}=-1\quad \varepsilon_{312}=1\\
			&\varepsilon_{213}=-1\quad\varepsilon_{321}=-1
		\end{split}
	\end{equation*}
	Y ahora sustituyendo:
	\begin{equation*}
		\begin{split}
			&=a_{11}a_{22}a_{33}+a_{12}a_{23}a_{31}+a_{13}a_{21}a_{32}\\
			&-(a_{11}a_{23}a_{32}+a_{12}a_{21}a_{33}+a_{13}a_{22}a_{31})
		\end{split}
	\end{equation*}
	Si ahora comparamos con el determinante de una $3 \times 3$:
	\begin{equation*}
		\begin{split}
			\det(3 \times 3)&=a_{11}a_{22}a_{33}+a_{12}a_{23}a_{31}+a_{13}a_{21}a_{32}\\
							&-(a_{13}a_{22}a_{31}+a_{12}a_{21}a_{33}+a_{11}a_{23}a_{32})
		\end{split}
	\end{equation*}
	Vemos que hemos obtenido el mismo resultado.
	\section*{Problema 4 [2 puntos]}
	Calcula
	\begin{equation*}
		\sum_{i=1}^{3}\sum_{j=1}^{3}\sum_{k=1}^{3}\epsilon_{ijk}\,\epsilon_{ijk}
	\end{equation*}
	\subsection*{Solución}
	Como $\varepsilon_{ijk}=1$ o $\varepsilon_{ijk}=-1$, $\varepsilon_{ijk}^{2}=1$:
	\begin{equation*}
		\begin{split}
			&= \sum_{i=1}^{3}\sum_{j=1}^{3}\varepsilon_{ij 1}^{2}+\varepsilon_{ij 2}^{2}+\varepsilon_{ij 3}^{2}\\
			&= \sum_{i=1}^{3}\varepsilon_{i 11}^{2}+\varepsilon_{i 12}^{2}+\varepsilon_{i 13}^{2}\\
			&+ \varepsilon_{i 21}^{2}+\varepsilon_{i 22}^{2}+\varepsilon_{i 23}^{2}\\
			&+ \varepsilon_{i 31}^{2}+\varepsilon_{i 32}^{2}+\varepsilon_{i 33}^{2}\\
			&= \sum_{i=1}^{3} \varepsilon_{i 12}^{2}+\varepsilon_{i 13}^{2}+\varepsilon_{i 21}^{2}\\
			&+ \varepsilon_{i 23}^{2}+\varepsilon_{i 31}^{2}+\varepsilon_{i 32}^{2}
		\end{split}
	\end{equation*}
	De nuevo, vamos a eliminar directamente aquellos elementos que vayan a ser $0$ una vez realidad la sustitución, obteniendo así:
	\begin{equation*}
		\begin{split}
			&= \varepsilon_{123}^{2}+\varepsilon_{132}^{2}\\
			&+ \varepsilon_{213}^{2}+\varepsilon_{231}^{2}\\
			&+ \varepsilon_{312}^{2}+\varepsilon_{321}^{2}
		\end{split}
	\end{equation*}
	Como estos elementos van a ser siempre $1 ^{2}$ o $(-1) ^{2}$, obtenemos
	\[
		= 6
	\]
\end{document}
