%! TEX root = FisicaII.tex

\documentclass{../FisicaII.tex}

\begin{document}
\begin{defin}
	Definimos la \textbf{fuerza de Coulomb} entre dos cargas, siendo una fija
	(foco) y otra la que se mueve (receptora).
	\begin{equation}
		\begin{split}
			\vec{F} = \frac{1}{4\pi \varepsilon_{0}} \frac{q_1q_2}{r^{2}}\vec{u}_{r}
			= k_{0} \frac{q_1q_2}{r^{2}}\vec{u}_{r}~(N)
		\end{split}
	\end{equation}
	Siendo $k_{0} = 9 \cdot 10^{9}~( \frac{Nm^{2}}{r^{2}})$, $\varepsilon_{0}$
	la permitividad eléctrica del vacío, y $\vec{u}_{r}$ el vector que va desde
	el foco al receptor.
\end{defin}
\begin{defin}
	El campo eléctrco es una region en el espacio alrededor de una carga en la que
	se pueda situar otra carga receptora para que esta sufra una fuerza
	repulsora o atractora.
	\begin{equation}
		\begin{split}
			\vec{E} = \frac{1}{4\pi \varepsilon_{0}} \frac{q}{r^{2}}\vec{u}_{r}
			~\left(\frac{N}{C}\right)
		\end{split}
	\end{equation}
	Podemos representar un campo eléctrico mediante punto-vector o líneas de
	fuerza.
\end{defin}
\begin{defin}
	Las líneas de fuerza es aquella que tienen como origen el foco, y su vector
	tangente es el campo eléctrico en cada uno de sus puntos.
\end{defin}
\subsection{Sistema de cargas puntuales}
Supongamos un conjunto de $N$  cargas $q_{1},\dots ,q_{n}$ y están situadas en
sus vectores $\vec{r}_{1},\dots ,\vec{r}_{n}$. La fuerza que actúa sobre una
carga receptora $Q$ cuyo vector posición es $\vec{r}_{p}$ es la suma vectorial
de las fuerzas creadas por el resto de las cargas:
\begin{equation}
	\begin{split}
		\vec{F}_{p} &= \sum_{i=1}^{N} \frac{1}{4\pi\varepsilon_{0}}
		\frac{q_{i}Q}{|r_{p}-r_{i}|^{3}}(r_{p}-r_{i}) ~(N)\\
					&= \frac{Q}{4\pi\varepsilon_{0}}\sum_{i=1}^{N}
					\frac{q_{i}}{|r_{p}-r_{i}|^{3}}(r_{p}-r_{i})~(N)\\
					&= Q \sum_{i=1}^{N} \vec{E}_{i}~(N)
	\end{split}
\end{equation}
\end{document}
