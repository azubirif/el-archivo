\documentclass[12pt,reqno]{article}

\usepackage{amsmath}
\usepackage{amssymb, physics}
\usepackage{bm}
\usepackage{enumerate}
\usepackage[shortlabels]{enumitem}
\setlist{topsep=1.5em, itemsep=1.5em}
\usepackage[exercisename=Problema,solutionname=Solución]{exercises}
\usepackage{hyperref}
\usepackage{graphicx}
\usepackage{mathtools}
\usepackage{wrapfig2}

\newcommand{\bbR}{\mathbb{R}}
\newcommand{\cC}{\mathcal{C}}
\newcommand{\cQ}{\mathcal{Q}}
\newcommand{\cS}{\mathcal{S}}
\newcommand{\dparcial}[2]{\frac{\partial#1}{\partial#2}}
\newcommand{\derivada}[2]{\frac{d#1}{d#2}}
\newcommand{\evaluar}[2]{\left.#1\right|_{#2}}
\newcommand{\exterior}{\text{ext }}
\newcommand{\figref}[1]{\text{fig. }\ref{#1}}
\newcommand{\frontera}{\partial}
\newcommand{\half}{\frac{1}{2}}
\newcommand{\interior}{\text{int }}
\newcommand{\sgn}{\text{sgn}}
\newcommand{\union}{\cup}

\title{Entrega 2}
%\date{\today}

\author{Alejandro Zubiri Funes}
%Si sois varias personas entonces quitad el símbolo % de las siguientes lineas y ponédselo a la anterior
%\author{Nombre1 Apellidos1\\
	%	Nombre2 Apellidos2\\
	%	Nombre3 Apellidos3\\
	%	Nombre4 Apellidos4}

\begin{document}
	\maketitle
	\section*{Problema 1}
	Sea
	\begin{equation*}
		f(x,y)=
		\left\{
		\begin{aligned}
			y\sin\frac{1}{x^2+y^2} &\qquad\text{si }(x,y)\in\bbR^2-\{(0,0)\}\\
			0 &\qquad\text{si }(x,y)=(0,0)
		\end{aligned}
		\right.
	\end{equation*}
	\begin{enumerate}[(a)]
		\item Halla los puntos en los que $f$ es continua.  [1 punto]
		\subsection*{Solución}
		La función no presenta problemas ni discontinuidades en ningún otro punto excepto $\textbf{x}=(0,0)$, donde podría llegar a ser discontinua. Para ello, vamos a tomar el límite de $(x,y)$ tendiendo a $(0,0)$:
		\[
			\lim_{(x,y) \to (0,0)} y \sin(\frac{1}{x^{2}+y^{2}}) 
		\]
		Como la función $sin$ está acotada entre $[-1,1]$, no tenemos que preocuparnos por el hecho de evaluar $\sin(\infty)$. Por tanto, estamos en una situación donde tenemos una constante por un valor que se acerca a $0$, y por tanto, el límite es $0$.\\
		Por tanto, como el límite y el valor de la función coinciden, es continua en ese punto, y por tanto lo es en todo $\mathbb{R}^{2}$. 
		\item Halla los puntos en los que existen las primeras derivadas parciales de $f$. [1 punto]
		\subsection*{Solución}
		Primer determinamos las respectivas derivadas parciales de la función:
		\begin{equation*}
			\pdv{f}{x}=\left\{\begin{aligned}
				\frac{-2xy}{(x^{2}+y^{2})^{2}}\cos \frac{1}{x^{2}+y^{2}}
				~si~(x,y)\neq(0,0)\\
				0~si~(x,y)=(0,0)
			\end{aligned}\right
		\end{equation*}
		\begin{equation*}
			\pdv{f}{y}= \left\{\begin{aligned}
			\sin(\frac{1}{x ^{2}+y ^{2}})- \frac{2y ^{2}}{(x ^{2}+y ^{2})^{2}}\cos(\frac{1}{x ^{2}+y ^{2}})~si~(x,y)\neq(0,0)\\
			0~si~(x,y)=(0,0)
		\end{aligned}\right	
		\end{equation*}
		Ahora se puede observar que ambas funciones están definidas para todo $\mathbb{R} ^{2}$. El único problema sería en $(x,y)=(0,0)$, pero hemos definido la función para que sea igual a $0$, por lo que existe en todos los puntos.   
		\newpage
		
		\item Halla los puntos en los que $f$ es diferenciable. [1 punto]
		\subsection*{Solución}
		Para que $f$ sea diferenciable en un punto $\textbf{x}$, las derivaras parciales de $f$ deben existir y ser continuas en una cierta bola $B_{r}(\textbf{x})$. Como sabemos que estas derivadas existen en todo $\mathbb{R} ^{2}$, falta ver si estas son continuas en $(0,0)$:
		\[
			\lim_{(x,y) \to (0,0)} \frac{-2xy}{(x ^{2}+y ^{2})^{2}}\cos(\frac{1}{x ^{2}+y ^{2}})
		\]
	Como $\cos$ está acotado, solo es necesario fijarse en el término de la izquierda. Para ello, vamos a comprobar el límite cuando se sigue el camino $y=mx:m \in \mathbb{R}$:
	\[
		=  \lim_{x \to 0} \frac{-2mx ^{2}}{x^{4}(1+m ^{2})^{2}}=
		\lim_{x \to 0} -\frac{2m}{x ^{2}(1+m ^{2})^{2}}=\infty 
	\]
	Como no coincide con el valor de la derivada en ese punto, no es continua, y por tanto $f$ es diferenciable en $\mathbb{R} ^{2}-\{ (0,0) \}$.   
	\item Halla, si existe, la derivada de $f$ respecto del vector $\bm v=\left(\frac{3}{5},\frac{4}{5}\right)$ en el punto $\bm a=(1,0)$. [1 punto]
		\subsection*{Solución}
		Nos interesa obtener lo siguiente:
		\[
			f'(\mathbf{a}, \mathbf{v})=\nabla f(\mathbf{a})\cdot \mathbf{v}
		\]
	Como hemos obtenido ambas derivadas parciales, podemos rápidamente evaluar ambas en el punto $\mathbf{a}=(1,0)$ para obtener que:
		\[
			\nabla f(1,0) = (0, \sin(1))
		\]
		Y ahora para obtener la derivada direccional correspondiente, simplemente calculamos el producto escalar de este vector con el vector $\textbf{v}$:
		\[
			f'(\mathbf{a}, \mathbf{v})=(0, \sin(1))\cdot (\frac{3}{5}, \frac{4}{5})=\frac{4}{5}\sin(1)
		\]
	\end{enumerate}
	\newpage
	
	\section*{Problema 2}
	Sea $f$ la función definida en todo $\bbR^2$ de la siguiente manera
	\begin{equation*}
		f(x,y)=
		\left\{
		\begin{aligned}
			\frac{x^3}{y-1} &\qquad\text{si $y\neq 1$}\\
			0 &\qquad\text{si }y=1
		\end{aligned}
		\right.
	\end{equation*}
	\begin{enumerate}[(a)]
		\item Estudia si $f$ es continua y diferenciable en $(0,1)$. [1 punto]
		\subsection*{Solución}
		Si tomamos el límite cuando $(x,y) \to (0,0)$, y tomamos el camino $y=1+x^{4}$, vemos que:
		\[
			\lim_{(x,y) \to (0,0)} \frac{x^{3}}{y-1}=\lim_{x \to 0}
			\frac{x^{3}}{1+x^{4}-1} = \lim_{x \to 0} \frac{1}{x} = \infty 
		\]
		Por tanto, la función no es continua en ese punto, y por tanto tampoco será diferenciable.
	\item Halla el vector unitario en cuya dirección es mínima la correspondiente derivada direccional de $f$ en el punto $(1,0)$. [1 punto]
		\subsection*{Solución}
		Teniendo en cuenta que buscamos un vector unitario, lo que buscamos es
		\[
			f'((1,0), \mathbf{u}) = \nabla f(1,0)\cdot \mathbf{u}=|\nabla f(1,0)| \cos \theta
		\]
		Por tanto, solo nos interesa el ángulo que formará este vector con el gradiente en dicho punto. Como el gradiente apuntará hacia donde más crece la función, el vector donde esta derivada direccional será mínima será cuando $\theta = \pi$, es decir, cuando el vector va en dirección contraria. Por tanto, este vector será
		\[
			\frac{1}{|\nabla f(1,0)|} \nabla f(1,0)
		\]
		Ahora calculamos las correspondientes derivadas parciales:
		\[
			\pdv{f}{x} =\left\{ \begin{matrix}
				\frac{3x^{2}}{y-1}~si~y\neq 1\\
				0~si~y=1
			\end{matrix}\right
		\]
		\[
			\pdv{f}{y} = \left\{ \begin{matrix}
				\frac{-x^{3}}{(y-1)^{2}}~si~y=1\\
				0~si~y=1
			\end{matrix}\right
		\]
		Ahora evaluamos en el punto para obtener
		\[
			\nabla f(1,0)=(-3,-1)
		\]
		y por tanto, el vector será este, dividido entre su módulo, y en dirección contraria:
		\[
			\boxed{
				\mathbf{u} = \frac{\sqrt{10}}{10}(3,1)
			}
		\]
\item Halla la ecuación del plano tangente a la gráfica de $f$ en el punto $(1,0)$. [1 punto]
		\subsection*{Solución}
		Si sustituimos en la función, obtenemos que el punto es $(1,0,-1)$. Ahora bien, como esta función permite obtener $z$ de forma explícita (es decir, podemos tener $z$ únicamente en un lado, y luego $x,y$ en el otro lado), podemos definir la función $g(x,y,z)=f(x,y)-z$, y ahora tomamos el gradiente y sustutimos, obteniendo:
		\[
			\nabla g(1,0,-1)=(-3,-1,-1)
		\]
		y por tanto el plano será de la forma
		\[
			-3x-y-z+c=0
		\]
		sustutimos el el punto y despejamos $c$, para obtener que
		\[
			\pi \equiv 3x+y+z-2=0
		\]
	\end{enumerate}
	
	\newpage
	
	\section*{Problema 3}
	Halla los planos tangentes a las siguientes superficies en los puntos indicados
	\begin{enumerate}[(a)]
		\item $z=x^2+y^2$ en $(3,1,10)$. [1 punto]
		\subsection*{Solución}
		Para proceder con estos problemas, despejaremos todos los elementos a un lado de la ecuación y tomaremos el gradiente. Para la primera superfície, obtenemos que el gradiente de nuestra función $f(x,y,z)=x^{2}+y^{2}-z$ es
		\[
			\nabla f=(2x,2y,-1)
		\]
		Sustutuimos el punto para obtener que el gradiente en ese punto es $(6,2,-1)$. Ahora planteamos el plano y sustutimos el valor del punto para despejar la constante, para obtener que el plano es
		\[
			6x+2y-z-10=0
		\]
		\item $x^2+(y-2)^2+2z^2=4$ en $(1,3,-1)$. [1 punto]
		\subsection*{Solución}
		De nuevo, pasamos los elementos a un lado y tomamos el gradiente, obteniendo así
		\[
			(2x,2y-4, 4z)
		\]
		Suistuimos para obtener $(2,2,-4)$. Ahora sustituimos el punto, obtenemos $c=-12$, y obtenemos
		\[
			x+y+2-6=0
		\]
	\item $yz=\ln(x+z)$ en $(0,0,1)$. [1 punto]
		\subsection*{Solución}
		Despejando los elementos, la función queda definida como $h(x,y,z)=\ln(x+z)-yz$. Tomamos el gradiente para obtener
		\[
			\nabla h(x,y,z)=(\frac{1}{x+z}, -z, \frac{1}{x+z}-y)
		\]
		Sustituimos para obtener que el gradiente en ese punto es $(1,-1,1)$. Ahora para obtener $c$, sustituimos el punto, y obtenemos que el plano es
		\[
			x-y+z-1=0
		\]
\end{enumerate}

	
	
	
	
	
	
	
	
	
	
	
	
	
	
	
	
	
	
	
	
	
	
	
	
	
	
	
	
	
	
	
	
	
	
	
	
	
	
	
	
	
	
	
	
	
	
	
	
	
	
	
	
	
	
	
	
	
	
	
	
	
	
	
	
	
	
	
	
	
	
	
	
	
	
	
	
	
\end{document}
