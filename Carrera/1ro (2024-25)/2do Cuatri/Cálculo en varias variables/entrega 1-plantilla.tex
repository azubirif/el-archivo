\documentclass[12pt,reqno]{article}


\usepackage[spanish,es-noquoting]{babel}
\usepackage{amsmath}
\usepackage{amssymb}
\usepackage{bm}
\usepackage{enumerate}
\usepackage[shortlabels]{enumitem}
\setlist{topsep=1.5em, itemsep=1.5em}
\usepackage[exercisename=Problema,solutionname=Solución]{exercises}
\usepackage{hyperref}
\usepackage{graphicx}
\usepackage{mathtools}
\usepackage{wrapfig2}

\newcommand{\bbR}{\mathbb{R}}
\newcommand{\cC}{\mathcal{C}}
\newcommand{\cQ}{\mathcal{Q}}
\newcommand{\cS}{\mathcal{S}}
\newcommand{\dparcial}[2]{\frac{\partial#1}{\partial#2}}
\newcommand{\derivada}[2]{\frac{d#1}{d#2}}
\newcommand{\evaluar}[2]{\left.#1\right|_{#2}}
\newcommand{\exterior}{\text{ext }}
\newcommand{\figref}[1]{\text{fig. }\ref{#1}}
\newcommand{\frontera}{\partial}
\newcommand{\grad}{\text{grad }}
\newcommand{\half}{\frac{1}{2}}
\newcommand{\interior}{\text{int }}
\newcommand{\norm}[1]{\left\|#1\right\|}
\newcommand{\sgn}{\text{sgn}}
\newcommand{\tr}{\text{Tr }}
\newcommand{\union}{\cup}

\title{Entrega 1}
\author{Alejandro Zubiri Funes}
\date{\today}

\begin{document}
	\maketitle
	\section*{Problema 1 [3 puntos]}
	Si $A=(a_{ij})$ y $S=(s_{ij})$ son matrices $n\times n$ tales que
	\begin{align*}
		a_{ji}&=-a_{ij}\\
		s_{ji}&=s_{ij}
	\end{align*}
	demuestra que
	\begin{equation*}
		\sum_{i=1}^{n}\sum_{j=1}^{n} a_{ij}\,s_{ij}=0.
	\end{equation*}
	
	\subsection*{Solución}
Si empezamos analizando las consecuencias de ambas restricciones, obtenemos
lo siguiente:
\begin{itemize}
	\item Para la primera matriz, si $a_{ij}=-a_{ji}$, tenemos que la matriz
		es \textbf{antisimétrica}, es decir, los elementos por debajo de la
		diagonal son los opuestos a los de por encima de la diagonal. Y respecto
		a la diagonal, se tiene que $a_{ii}=-a_{ii}$, por lo que la única
		posibilidad es una diagonal llena de ceros.
	\item Las propiedades de la segunda matriz son análogas, excepto que es
		\textbf{simétrica}, es decir, los elementos por debajo de la diagonal son
		los mismos a los de por encima de la diagonal. Con esto, obtenemos
		las siguientes matrices:
\end{itemize}
\[A=
	\begin{pmatrix} 
		0 & a_{12} & \dots & a_{1n}\\
		-a_{12} & 0 & \dots & a_{2n}\\
		\dots &\dots &\dots &\dots \\
		-a_{1n}&-a_{2n} & \dots & 0
	\end{pmatrix} 
\]
\[S=
	\begin{pmatrix} 
		0 & a_{12} & \dots & a_{1n}\\
		a_{12} & 0 & \dots & a_{2n}\\
		\dots &\dots &\dots &\dots \\
		a_{1n}&a_{2n} & \dots & 0
	\end{pmatrix} 
\]
Ahora podemos empezar a desarrollar el sumatorio. Para empezar, todo producto de
la forma $a_{ii}\cdot s_{ij}$ o $a_{ij}\cdot s_{ii}$ será $0$ debido a las
propiedades explicadas anteriormente. Respecto al resto de elementos, al ser
una matrix $n \times n$, podemos reescribir los elementos del sumatorio como
los siguientes pares:
\[
	a_{ij}\cdot s_{ij} + a_{ji}\cdot s_{ji}
\]
Si ahora sustutuimos las restricciones iniciales:
\[
	a_{ij}\cdot s_{ij}-a_{ij}\cdot s_{ij}=0
\]
Como podemos realizar esta sustitución para cada elemento del sumatorio, la suma total es $0$. 
	\newpage
	
	\section*{Problema 2 [1'5 puntos + 1'5 puntos]}
	En las siguientes expresiones el rango de definición de los índices es $\{1,2,\ldots,n\}$, además se usa el criterio de suma de Einstein. Llega, en cada caso, a una expresión algebraica más sencilla.
	\begin{enumerate}[(a)]
		\item $\delta_{ii}$.
		\subsection*{Solución}
		Haciendo esto y lo otro...
		\item $\delta_{ij}\,\delta_{jk}$.
		\subsection*{Solución}
		Haciendo esto y lo otro...
	\end{enumerate}
	
	\section*{Problema 3 [2 puntos]}
	Expande el siguiente sumatorio
	\begin{equation*}
		\sum_{i=1}^{3}\sum_{j=1}^{3}\sum_{k=1}^{3}\epsilon_{ijk}\,a_{1i}\,a_{2j}\,a_{3k}
	\end{equation*}
	después da el valor correspondiente a cada $\epsilon$ y comprueba que obtienes el determinante de la matriz $(a_{ij})$.
	\subsection*{Solución}
	Haciendo esto y lo otro...
	
	\section*{Problema 4 [2 puntos]}
	Calcula
	\begin{equation*}
		\sum_{i=1}^{3}\sum_{j=1}^{3}\sum_{k=1}^{3}\epsilon_{ijk}\,\epsilon_{ijk}
	\end{equation*}
	\subsection*{Solución}
	Haciendo esto y lo otro...
	
	
	
	
	
	
	
	
	
	
	
	
	
	
	
	
	
	
	
	
	
	
	
	
	
	
	
	
	
	
	
	
	
	
	
	
	
	
	
	
	
	
	
	
	
	
	
	
	
	
	
	
	
	
	
\end{document}
