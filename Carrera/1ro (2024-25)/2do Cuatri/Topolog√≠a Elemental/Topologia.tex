\documentclass{article}
\author{Alejandro Zubiri}
\title{Topología Elemental}

\renewcommand*\contentsname{Índice}

\usepackage[margin=1.1in]{geometry}
\usepackage{amsmath, physics, amsthm, amsfonts, mdframed}

\newmdtheoremenv{teorema}{Teorema}
\newmdtheoremenv{defin}{Definición}
\hbadness = 99999
\newcommand{\R}{\mathbb{R}}

\begin{document}
\maketitle
\tableofcontents
\pagebreak
\section{Nociones básicas}
Cuando definimos algo, tiene que estar definido de forma que cualquier persona
esté de acuerdo con dicha definición.
\begin{defin}
	Un conjunto se puede definir por su \textbf{extensión}, mencionando todos
	sus elementos, o por \textbf{compresión}, defininiendo la regla que todos
	los elementos del conjunto deben cumplir.
	\begin{itemize}
		\item Extensión:\\
			\begin{equation}
				\begin{split}
					S = \{ 1,2,3,\dots  \}
				\end{split}
			\end{equation}
		\item Comprensión:\\
			\begin{equation}
				\begin{split}
					S = \{ x \in \mathbb{N} \}
				\end{split}
			\end{equation}
	\end{itemize}
	Denotamos los conjuntos por letras mayúsculas, y sus elementos por letras
	minúsculas.\\
	Si $x$ es un elemento del conjunto $S$, decimos que $a \in S$, y si no
	pertenece, $a \notin S$
\end{defin}
\begin{defin}
	El \textbf{cardinal} de un conjunto es el número de elementos del conjunto,
	denotado por $\#S$. 
\end{defin}
Dados dos conjuntos $A$ y $B$, decimos que $A$ es un subconjunto de $B$ si y solo
si
\begin{equation}
	\begin{split}
		\forall x \in A, x \in B \implies A \subset B
	\end{split}
\end{equation}
Sino, decimos que $A \not\subset B$.
\begin{defin}
	Decimos que $A$ es subconjunto de $B$ si
	\begin{equation}
		\begin{split}
			A \subset B \iff (a \in A \implies a \in B)
		\end{split}
	\end{equation}
\end{defin}
Un ejemplo de conjuntos es el conjunto vacío:
\begin{equation}
	\begin{split}
		\phi / \# \phi = 0
	\end{split}
\end{equation}
\begin{defin}
	Decimos que dos conjuntos $A$ y $B$ son iguales si y solo si
	\begin{equation}
		\begin{split}
			A \subset B \wedge B \subset A
		\end{split}
	\end{equation}
\end{defin}
\subsection{Operaciones con conjuntos}
\begin{defin}
	La unión $S$ de dos conjuntos $A$ y $B$ es
	\begin{equation}
		\begin{split}
			S = A \cup B = \{x / x \in A \vee x \in B \}
		\end{split}
	\end{equation}
\end{defin}
\begin{defin}
	La intersección $S$ de dos conjuntos $A$ y $B$ es
	\begin{equation}
		\begin{split}
			S = A \cap B = \{ x / x \in A \wedge x \in B \}
		\end{split}
	\end{equation}
\end{defin}
\begin{defin}
	Definimos la diferencia $S$ de $A$ menos $B$ tal que
	\begin{equation}
		\begin{split}
			S = A - B = \{ x / x \in A \wedge x \notin B  \}
		\end{split}
	\end{equation}
\end{defin}
\begin{defin}
	La diferencia simétrica entre $E$ y $A$ es
	\begin{equation}
		\begin{split}
			A \Delta B = (A-B)\cup (B-A)
		\end{split}
	\end{equation}
\end{defin}

\begin{defin}
	Definimos el complemento $S^{c}$ de un conjunto $S$ como
	\begin{equation}
		\begin{split}
			S \cup S^{c} &= E\\
			S \cap S^{c} &= \phi
		\end{split}
	\end{equation}
	Siendo $E$ el conjunto total. 
\end{defin}
\section{Tablas de verdad}
Una tabla de verdad nos permite analizar como se comportan dos proposiciones:
\begin{table}[h]
	\centering

	\begin{tabular}{c|c|c|c}
		$p$ & $q$ & $p\vee q$\\
		\hline
		$V$ & $V$ & $V$\\
		$V$ & $F$ & $V$\\
		$F$ & $V$ & $V$\\
		$F$ & $F$ & $F$\\  
	\end{tabular}
\end{table}
\begin{table}[h]
	\centering

	\begin{tabular}{c|c|c}
		$p$ & $q$ & $p\wedge q$\\
		\hline
		$V$ & $F$ &$F$\\
		$F$&$V$&$F$\\
		$F$ & $F$ & $F$\\
		$V$ & $V$ & $V$
	\end{tabular}
\end{table}
\begin{table}[h]
	\centering

	\begin{tabular}{c|c|c}
		$p$ & $q$ & $p \implies q$\\
		\hline
		$V$ & $V$ & $V$\\
		$V$&$F$&$F$\\
		$F$&$V$&$V$\\
		$F$&$F$&$V$
		
	\end{tabular}
\end{table}
Se puede deducir que $p \implies q \iff \neg p \vee q$ 
\end{document}
