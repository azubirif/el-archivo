%! TEX root = Topologia.tex

\documentclass{../Topologia.tex}

\begin{document}

\section{Conjuntos}
Cuando definimos algo, tiene que estar definido de forma que cualquier persona
esté de acuerdo con dicha definición.
\begin{defin}
	Un conjunto se puede definir por su \textbf{extensión}, mencionando todos
	sus elementos, o por \textbf{compresión}, defininiendo la regla que todos
	los elementos del conjunto deben cumplir.
	\begin{itemize}
		\item Extensión:\\
			\begin{equation}
				\begin{split}
					S = \{ 1,2,3,\dots  \}
				\end{split}
			\end{equation}
		\item Comprensión:\\
			\begin{equation}
				\begin{split}
					S = \{ x \in \mathbb{N} \}
				\end{split}
			\end{equation}
	\end{itemize}
	Denotamos los conjuntos por letras mayúsculas, y sus elementos por letras
	minúsculas.\\
	Si $x$ es un elemento del conjunto $S$, decimos que $a \in S$, y si no
	pertenece, $a \notin S$. Es importante tener en cuenta que, a menos que se
	especifique, el orden de los elementos de un conjunto es irrelevante, solo nos
	interesan sus elementos. Para especificar orden, podemos utilizar $(a,b)$,
	que se define como \textbf{par ordenado}, tal que
	\[
		(a,b) = (c,d) \iff a=c \wedge b = d
	\]
\end{defin}

\begin{defin}
	El \textbf{cardinal} de un conjunto es el número de elementos del conjunto,
	denotado por $\#S$. 
\end{defin}
Dados dos conjuntos $A$ y $B$, decimos que $A$ es un subconjunto de $B$ si y solo
si
\begin{equation}
	\begin{split}
		\forall x \in A, x \in B \implies A \subset B
	\end{split}
\end{equation}
Sino, decimos que $A \not\subset B$.
\begin{defin}
	Decimos que $A$ es subconjunto de $B$ si
	\begin{equation}
		\begin{split}
			A \subset B \iff (a \in A \implies a \in B)
		\end{split}
	\end{equation}
\end{defin}
Un ejemplo de conjuntos es el conjunto vacío:
\begin{equation}
	\begin{split}
		\phi / \# \phi = 0
	\end{split}
\end{equation}
\begin{defin}
	Decimos que dos conjuntos $A$ y $B$ son iguales si y solo si
	\begin{equation}
		\begin{split}
			A \subset B \wedge B \subset A
		\end{split}
	\end{equation}
\end{defin}
\subsection{Operaciones con conjuntos}
\begin{defin}
	La unión $S$ de dos conjuntos $A$ y $B$ es
	\begin{equation}
		\begin{split}
			S = A \cup B = \{x / x \in A \vee x \in B \}
		\end{split}
	\end{equation}
\end{defin}
\begin{defin}
	La intersección $S$ de dos conjuntos $A$ y $B$ es
	\begin{equation}
		\begin{split}
			S = A \cap B = \{ x / x \in A \wedge x \in B \}
		\end{split}
	\end{equation}
\end{defin}
\begin{defin}
	Definimos la diferencia $S$ de $A$ menos $B$ tal que
	\begin{equation}
		\begin{split}
			S = A - B = \{ x / x \in A \wedge x \notin B  \}
		\end{split}
	\end{equation}
\end{defin}
\begin{defin}
	La diferencia simétrica entre $E$ y $A$ es
	\begin{equation}
		\begin{split}
			A \Delta B = (A-B)\cup (B-A)
		\end{split}
	\end{equation}
\end{defin}

\begin{defin}
	Definimos el complemento $S^{c}$ de un conjunto $S$ como
	\begin{equation}
		\begin{split}
			S \cup S^{c} &= E\\
			S \cap S^{c} &= \phi
		\end{split}
	\end{equation}
	Siendo $E$ el conjunto total. 
\end{defin}
\begin{defin}
	Definimos el producto cartesiano entre $A$ y $B$ como
	\[
		A \times B = \{ (a,b): a \in A \wedge b \in B \}
	\]
	Ejemplos:
	\begin{itemize}
		\item $\emptyset \times B = \emptyset$
		\item $A \times B \neq B \times A$ (ya que son pares con órdenes diferentes)
	\end{itemize}
\end{defin}

\begin{prop}
	$A\cap (B\cup C) = (A\cap B)\cup (A\cap C)$ 
\end{prop}
\begin{proof}[Demostración]
	Sea $x \in A\cap (B\cup C) \iff x \in A \wedge x \in B \cup C$
	\begin{equation}
		\begin{split}
			&\iff x \in A \wedge (x \in B \vee x \in C)\\
			&\iff (x\in A \wedge x \in B) \vee (x \in A \wedge x \in C)\\
			&\iff x \in A\cap B \vee x \in A \cap C\\
			&\iff x \in (A\cap B) \cup (A\cap C)
		\end{split}
	\end{equation}
\end{proof}
\begin{prop}
	$(A\cup B)^{c} = A^{c} \cup B^{c}$ 
\end{prop}
\begin{proof}[Demostración]
	\begin{equation}
		\begin{split}
			x \in A^{c} \cup B^{c} &\iff x \in A^{c} \vee x \in B^{c}\\
								   &\iff x \notin A \vee x \notin B\\
								   &\iff x \notin A\cap B\\
								   &\iff x \in (A\cap B)^{c}
		\end{split}
	\end{equation}
\end{proof}
\section{Tablas de verdad}
Una tabla de verdad nos permite analizar como se comportan dos proposiciones:
\begin{table}[h]
	\centering

	\begin{tabular}{c|c|c|c}
		$p$ & $q$ & $p\vee q$\\
		\hline
		$V$ & $V$ & $V$\\
		$V$ & $F$ & $V$\\
		$F$ & $V$ & $V$\\
		$F$ & $F$ & $F$\\  
	\end{tabular}
\end{table}
\begin{table}[h]
	\centering

	\begin{tabular}{c|c|c}
		$p$ & $q$ & $p\wedge q$\\
		\hline
		$V$ & $F$ &$F$\\
		$F$&$V$&$F$\\
		$F$ & $F$ & $F$\\
		$V$ & $V$ & $V$
	\end{tabular}
\end{table}
\begin{table}[h]
	\centering

	\begin{tabular}{c|c|c}
		$p$ & $q$ & $p \implies q$\\
		\hline
		$V$ & $V$ & $V$\\
		$V$&$F$&$F$\\
		$F$&$V$&$V$\\
		$F$&$F$&$V$
		
	\end{tabular}
\end{table}
Se puede deducir que $p \implies q \iff \neg p \vee q$. Con esto, también podemos
deducir que
\[
	(p \implies q) \iff (\neg q \implies \neg p)
\]
\begin{defin}
	Sea $S\subset \R$, una función $f:S\to \R$ es continua en $a\subset S$ si:
	\begin{equation}
		\begin{split}
			\forall \varepsilon > 0 \exists \delta > 0 /
			|f(x)-f(a)|<\varepsilon \implies |x-a| < \delta
		\end{split}
	\end{equation}
\end{defin}
\begin{defin}
	Sea $S \subset \R$. $S$ es abierto si:
	\begin{equation}
		\begin{split}
			\forall x \in S \exists I \subset S / I =
			(x-\delta, x+\delta) / \delta > 0
		\end{split}
	\end{equation}
\end{defin}
\begin{prop}
	La unión de abiertos es un abierto.
\end{prop}
\begin{proof}[Demostración]
	Sea $S_{i}$ cada conjunto abierto. Sabemos que
	\begin{equation}
		\begin{split}
			\forall x \in S_{i} \exists \delta > 0 / (x-\delta, x+\delta) \subset S
		\end{split}
	\end{equation}
	Sea $U$ la unión de los conjuntos:
	\begin{equation}
		\begin{split}
			U = S_{1} \cup S_{2}\dots \cup S_{n} = 
			\{ x / x \in S_{1} \vee \dots \vee x \in S_{n} \}
		\end{split}
	\end{equation}
	Sabemos que para cada punto $x \exists \delta > 0 / (x-\delta,x+\delta) \in S_{i}$.
	Por tanto, estos subintervalos estarán contenidos en la unión, y por tanto
	esta es abierta.
\end{proof}
\begin{prop}
	La intersección finita de abiertos es abierta
\end{prop}
\begin{proof}[Demostración]
	Vamos a definir dos casos:
	\begin{itemize}
		\item \textbf{Caso 1}: La intersección es $\emptyset$. 
		Como sabemos que $\emptyset$ es abierto, se cumple.
		\item \textbf{Caso 2}: La intersección noe es $\emptyset$
		La intersección estará formada por una serie de conjuntos no nulos que
		sabemos que contienen intervalos abiertos $\forall x$. Sea $\delta
		 / \delta = min(\delta_{1}, \dots ,\delta_{n})$. Como este $\delta$
		 es el más pequeño, estará contenido en todos los abiertos para
		 todos los puntos, y por tanto lo estará también en la intersección.
	\end{itemize}
\end{proof}
\subsection{Continuidad por conjuntos abiertos}
Vamos a definir el concepto de preimagen:
\begin{defin}
	Sea $f:D \to C$ una función y $S\subset C$. La preimagen de $S$ bajo $f$,
	escrita como $f^{-1}(S)$ es el subconjunto de $D$ definido como:
	\begin{equation}
		\begin{split}
			f^{-1}(S) = \{ x \in D / f(x) \in S \}
		\end{split}
	\end{equation}
\end{defin}

Sea $f:S \to T / U,V \subset T$

\end{document}
