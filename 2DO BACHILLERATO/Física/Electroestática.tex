\documentclass{article}

\usepackage{amsmath}
\usepackage{amsthm}

\newtheorem*{Conductor*}{Definición}
\newtheorem*{Insulador*}{Definición}
\newtheorem*{superposición*}{Definición}

\title{Electroestática}
\author{Alejandro Zubiri Funes}
\date{}

\begin{document}
\maketitle
\section*{Resumen}
\begin{itemize}
    \item Permitividad eléctrica del espacio: $\epsilon_0= 8.854\cdot 10^{-12}\:(\frac{C^2}{N\cdot m^2})$ 
    \item Constante de Coulomb: $K=9\cdot 10^9= \frac{1}{4\pi \epsilon_0}\:( \frac{N\cdot m^2}{C^2})$
    \item $1\:(C)\approx-q_e\cdot 6\cdot 10^{18}$ 
    \item Carga de un electrón: $q_e=-1.6\cdot 10^-19\:(C)$
    \item Carga de un protón: $q_p=-q_e$
    \item Campo eléctrico en un punto: $\mathbf{E}=K \frac{Q}{r^2}\hat{\mathbf{r}}\:(\frac{N}{C})$
    \item Ley de Coulomb: $\mathbf{F}=q\mathbf{E}=K \frac{qQ}{r^2}\hat{\mathbf{r}}\:(N)$
    \item Potencial eléctrico: $V=K \frac{Q}{r}\:(\frac{J}{C})$
    \item Energía potencial eléctria: $P=qV\:(J)$
    \item Trabajo por \textbf{fuerza externa}: $W_{ext}=q(\Delta V)=q(V_f-V_0)\:(J)$
    \item Trabajo por \textbf{campo}: $W_{campo}=-W_{ext}=-q(\Delta V)=-q(V_f-V_0)\:(J)$
    \item $A= \frac{C}{s}$
    \item $C\cdot V=J$
\end{itemize}
\pagebreak

\section*{Definiciones}
\begin{Conductor*}[Conductor]
    Material que permite que las cargas eléctricas se muevan con gran facilidad por él.
\end{Conductor*}

\begin{Insulador*}[Insulador]
    Material que dificulta el movimiento de las cargas eléctricas.
\end{Insulador*}

\section*{Ley de Coulomb}
La ley de Coulomb establece que la fuerza entre dos cargas puntuales es la siguiente:
\[
    \mathbf{F}=K \frac{q_1 q_2}{r^2}\hat{\mathbf{r}}\:(N)
\]
donde $\hat{\mathbf{r}}$ es el vector unitario al que apuntará cada fuerza. En función de los signos de las cargas, estas se \textbf{atraerán} o se
\textbf{repelerán}.\\
Esta ley sigue el \textbf{principio de superposición de fuerzas}.
\begin{superposición*}[Principio de superposición de fuerzas]
    La fuerza ejercida simultáneamente en una carga por un número $n$ de cargas es la suma vectorial de las fuerzas que ejercerían
    \textbf{individualmente} cada carga.
\end{superposición*}
Esta ley solo debería ser usada en el \textbf{vacío}, ya que la presencia de un material intermediario cambiaría la fuerza que actua en
cada carga, ya que parte de la carga se induce en el material intermediario.
\end{document}