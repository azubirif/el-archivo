\documentclass{article}

\usepackage{tikz}
\usepackage{amsmath}
\usepackage{bm}

\title{Sistemas de coordenadas}
\author{Alejandro Zubiri Funes}
\date{}

\begin{document}
\maketitle
Los sistemas de coordenadas son utilizados para describir la posición y otras propiedades del espacio de un objeto. Existen diferentes tipos de sistemas,
variando en función de la dimensión del espacio a tratar y el tipo de movimiento.

\subsection*{Sistemas en 3D}
Por convención, los sistemas en 3 dimensiones utilizan los siguientes ejes:


\tikzset{every picture/.style={line width=0.75pt}} %set default line width to 0.75pt        

\begin{tikzpicture}[scale = 0.5, every node/.style={scale=0.8}, x=0.75pt,y=0.75pt,yscale=-1,xscale=1]
%uncomment if require: \path (0,300); %set diagram left start at 0, and has height of 300

%Shape: Axis 2D [id:dp7216381744033507] 
\draw  (231,174.7) -- (394,174.7)(247.3,28) -- (247.3,191) (387,169.7) -- (394,174.7) -- (387,179.7) (242.3,35) -- (247.3,28) -- (252.3,35)  ;
%Straight Lines [id:da5922809991261604] 
\draw    (247.3,174.7) -- (153.41,268.59) ;
\draw [shift={(152,270)}, rotate = 315] [color={rgb, 255:red, 0; green, 0; blue, 0 }  ][line width=0.75]    (10.93,-4.9) .. controls (6.95,-2.3) and (3.31,-0.67) .. (0,0) .. controls (3.31,0.67) and (6.95,2.3) .. (10.93,4.9)   ;

% Text Node
\draw (135,275.4) node [anchor=north west][inner sep=0.75pt]    {$x$};
% Text Node
\draw (241,8.4) node [anchor=north west][inner sep=0.75pt]    {$z$};
% Text Node
\draw (399,166.4) node [anchor=north west][inner sep=0.75pt]    {$y$};


\end{tikzpicture}

Donde el eje $\mathrm{z}$ representaría la ``altura``.

\pagebreak

\subsection*{Sistema cartesiano}
El sistemas de coordenadas cartesiano describe la posición de un objeto mediante la distancia que tiene este al eje de coordenadas en cada eje.
Una posición se representaría de la siguiente forma:
\[
    \mathbf{r}=(x_r,y_r,z_r)
\]


\tikzset{every picture/.style={line width=0.75pt}} %set default line width to 0.75pt        

\begin{tikzpicture}[scale = 0.5, every node/.style={scale=0.8}, x=0.75pt,y=0.75pt,yscale=-1,xscale=1]
%uncomment if require: \path (0,300); %set diagram left start at 0, and has height of 300

%Shape: Axis 2D [id:dp7216381744033507] 
\draw  (231,174.7) -- (394,174.7)(247.3,28) -- (247.3,191) (387,169.7) -- (394,174.7) -- (387,179.7) (242.3,35) -- (247.3,28) -- (252.3,35)  ;
%Straight Lines [id:da5922809991261604] 
\draw    (247.3,174.7) -- (153.41,268.59) ;
\draw [shift={(152,270)}, rotate = 315] [color={rgb, 255:red, 0; green, 0; blue, 0 }  ][line width=0.75]    (10.93,-4.9) .. controls (6.95,-2.3) and (3.31,-0.67) .. (0,0) .. controls (3.31,0.67) and (6.95,2.3) .. (10.93,4.9)   ;
%Straight Lines [id:da40745792052990915] 
\draw  [dash pattern={on 4.5pt off 4.5pt}]  (214,207.6) -- (294,207.6) ;
%Straight Lines [id:da1967032713930439] 
\draw  [dash pattern={on 4.5pt off 4.5pt}]  (327.3,174.7) -- (294,207.6) ;
%Straight Lines [id:da053490985236124944] 
\draw  [dash pattern={on 4.5pt off 4.5pt}]  (294,207.6) -- (294,131.6) ;
%Shape: Circle [id:dp5884694882138559] 
\draw  [draw opacity=0][fill={rgb, 255:red, 0; green, 0; blue, 0 }  ,fill opacity=1 ] (291.6,129.2) .. controls (291.6,127.87) and (292.67,126.8) .. (294,126.8) .. controls (295.33,126.8) and (296.4,127.87) .. (296.4,129.2) .. controls (296.4,130.53) and (295.33,131.6) .. (294,131.6) .. controls (292.67,131.6) and (291.6,130.53) .. (291.6,129.2) -- cycle ;

% Text Node
\draw (135,275.4) node [anchor=north west][inner sep=0.75pt]    {$x$};
% Text Node
\draw (241,8.4) node [anchor=north west][inner sep=0.75pt]    {$z$};
% Text Node
\draw (399,166.4) node [anchor=north west][inner sep=0.75pt]    {$y$};
% Text Node
\draw (238,213.4) node [anchor=north west][inner sep=0.75pt]    {$y_{r}$};
% Text Node
\draw (324,185.4) node [anchor=north west][inner sep=0.75pt]    {$x_{r}$};
% Text Node
\draw (303,142.4) node [anchor=north west][inner sep=0.75pt]    {$z_{r}$};
% Text Node
\draw (298,120.4) node [anchor=north west][inner sep=0.75pt]    {$\mathbf{r}$};


\end{tikzpicture}

\subsection*{Sistema esférico}
El sistema esférico parte de un radio (distancia \textbf{total} al eje de coordenadas) y dos ángulos respecto del eje $\mathrm{x}$ y el eje $\mathrm{z}$.
Este sistema es muy conveniente para sistemas donde haya un objeto rotanto, como una órbita.


\tikzset{every picture/.style={line width=0.75pt}} %set default line width to 0.75pt        

\begin{tikzpicture}[scale = 0.5, every node/.style={scale=0.8}, x=0.75pt,y=0.75pt,yscale=-1,xscale=1]
%uncomment if require: \path (0,300); %set diagram left start at 0, and has height of 300

%Shape: Axis 2D [id:dp7216381744033507] 
\draw  (231,174.7) -- (394,174.7)(247.3,28) -- (247.3,191) (387,169.7) -- (394,174.7) -- (387,179.7) (242.3,35) -- (247.3,28) -- (252.3,35)  ;
%Straight Lines [id:da5922809991261604] 
\draw    (247.3,174.7) -- (153.41,268.59) ;
\draw [shift={(152,270)}, rotate = 315] [color={rgb, 255:red, 0; green, 0; blue, 0 }  ][line width=0.75]    (10.93,-4.9) .. controls (6.95,-2.3) and (3.31,-0.67) .. (0,0) .. controls (3.31,0.67) and (6.95,2.3) .. (10.93,4.9)   ;
%Straight Lines [id:da40745792052990915] 
\draw  [dash pattern={on 4.5pt off 4.5pt}]  (247.3,174.7) -- (294,207.6) ;
%Straight Lines [id:da1967032713930439] 
\draw  [dash pattern={on 4.5pt off 4.5pt}]  (294,131.6) -- (294,207.6) ;
%Straight Lines [id:da053490985236124944] 
\draw  [dash pattern={on 4.5pt off 4.5pt}]  (247.3,174.7) -- (294,131.6) ;
%Shape: Circle [id:dp5884694882138559] 
\draw  [draw opacity=0][fill={rgb, 255:red, 0; green, 0; blue, 0 }  ,fill opacity=1 ] (291.6,129.2) .. controls (291.6,127.87) and (292.67,126.8) .. (294,126.8) .. controls (295.33,126.8) and (296.4,127.87) .. (296.4,129.2) .. controls (296.4,130.53) and (295.33,131.6) .. (294,131.6) .. controls (292.67,131.6) and (291.6,130.53) .. (291.6,129.2) -- cycle ;
%Shape: Arc [id:dp5503445102030255] 
\draw  [draw opacity=0] (276,196.06) .. controls (268.12,206.32) and (255.17,212.25) .. (241.41,210.53) .. controls (233.34,209.52) and (226.19,206.03) .. (220.62,200.92) -- (246.08,173.13) -- cycle ; \draw   (276,196.06) .. controls (268.12,206.32) and (255.17,212.25) .. (241.41,210.53) .. controls (233.34,209.52) and (226.19,206.03) .. (220.62,200.92) ;  
%Shape: Arc [id:dp6047064650476885] 
\draw  [draw opacity=0] (248.44,127.91) .. controls (249.98,127.95) and (251.54,128.07) .. (253.1,128.26) .. controls (263.94,129.62) and (273.44,134.57) .. (280.57,141.78) -- (247.3,174.7) -- cycle ; \draw   (248.44,127.91) .. controls (249.98,127.95) and (251.54,128.07) .. (253.1,128.26) .. controls (263.94,129.62) and (273.44,134.57) .. (280.57,141.78) ;  

% Text Node
\draw (135,275.4) node [anchor=north west][inner sep=0.75pt]    {$x$};
% Text Node
\draw (241,8.4) node [anchor=north west][inner sep=0.75pt]    {$z$};
% Text Node
\draw (399,166.4) node [anchor=north west][inner sep=0.75pt]    {$y$};
% Text Node
\draw (298,120.4) node [anchor=north west][inner sep=0.75pt]    {$\mathbf{r}$};
% Text Node
\draw (245,214.4) node [anchor=north west][inner sep=0.75pt]    {$\Phi _{r}$};
% Text Node
\draw (252,106.4) node [anchor=north west][inner sep=0.75pt]    {$\theta _{r}$};
% Text Node
\draw (272.65,151.55) node [anchor=north west][inner sep=0.75pt]    {$r$};


\end{tikzpicture}

\subsection*{Aplicación del sistema esférico}
Supongamos que tenemos un objeto en la posición $\mathbf{r}=(r, \psi, \theta)$. Debido a que está orbitando un objeto, su ecuación de movimiento es la siguiente:
\[
    \ddot{\mathbf{r}}=g(r)\hat{\mathbf{r}}
\]
donde $\hat{\mathbf{r}}$ es el \textbf{vector unitario}, que apunta al objeto al que se orbita.
\end{document}