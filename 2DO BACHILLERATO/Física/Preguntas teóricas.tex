\documentclass{article}
\title{Preguntas de teoría}
\author{Alejandro Zubiri Funes}
\date{}

\begin{document}
\maketitle
\subsection*{P: ¿Cuáles son las interacciones fundamentales de la naturaleza?}
R: La interacción nuclear fuerte, la débil, la electromagnética, y la gravitatoria.

\subsection*{P: ¿Qué nombres se les dan a las observaciones que apoyan la teoría del "big-bang"?}
R: La radiación de fondo de microondas y el efecto Doppler relativista.

\subsection*{P: Describe con una frase el efecto Doppler relativista.}
R: Cambio de la frecuencia de la luz que procede de una fuente en movimiento relativo respecto al observador.

\subsection*{P: ¿Qué experimentos o sucesos no pueden ser explicados por la física clásica?}
R: La precesión de Mercurio, el experimento Michelson-Morley, el experimento de doble rendija, la discontinuidad del espectro atómico, el efecto fotoeléctrico, radiación del cuerpo negro.

\subsection*{P: ¿Qué tipo de desintegración radioactiva se produce en el Carbono 14?}
R: Radiación beta $(\beta)$

\subsection*{P: ¿Qué interacción mantienen los protones cercanos unos de los otros dentro del núcleo atómico?}
R: La interacción nuclear fuerte.

\subsection*{P: Explica el concepto de período de semidesintegración.}
R: Es el tiempo que debe pasar para que la actividad de una muestra radioactiva se reduzca a la mitad.

\end{document}