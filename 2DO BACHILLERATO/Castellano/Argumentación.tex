\documentclass{article}
\title{Argumentación}
\date{}
\author{}

\begin{document}
\maketitle

\section*{Tipos de argumentos}
\begin{itemize}
    \item \textbf{Factuales}: datos verídicos y objetivos, como fechas, cirftas, etc.
    \item \textbf{Tópicos}: basados en valores de sociedad. Opiniones de la mayoría, sabiduría popular\dots
    \item \textbf{De autoridad}: argumentos de personas con conocimiento, como expertos en la materia o similares.
    \item \textbf{Ejemplificativos}: casos específicos que apoyan la tesis.
    \item \textbf{Analógicos}: relacionar situaciones similares para hacer analogías con las que apoyar el tema.
    \item \textbf{Afectivos}: tratan de utilizar las emociones del lector para que apoye nuestra tesis.
\end{itemize}

\section*{Estructura de una argumentación}
\subsection{Introducción y tesis}
Introducción del tema, y damos \textbf{claramente} nuestra opinión sobre el tema a desarrollar.

\subsection{Argumentos}
Desarrollaremos \textbf{dos argumentos de diferente tipo} a favor de nuestra opinión, utilizando los tipos de argumentos explicados con anterioridad.

\subsection{Contraargumento}
Explicaremos un posible argumento \textbf{en contra} del nuestro, y lo refutaremos.

\subsection{Conclusión}
Resumiremos brevemente nuestros argumentos y tesis, y existe la posibilidad de finalizar con preguntas abiertas, citas\dots

\end{document}