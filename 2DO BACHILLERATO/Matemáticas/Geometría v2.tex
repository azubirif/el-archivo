\documentclass{article}
\title{Geometría - Versión actualizada}
\author{Alejandro Zubiri Funes}

\usepackage{amsmath}
\usepackage{amsthm}

\newtheorem*{normal*}{Definición}

\date{}

\begin{document}
\maketitle
\section*{Terminología}
\begin{itemize}
    \item $\vec{V}$: vector.
    \item $\vec{u}$: vector director.
    \item $|u|=\sqrt{u_1^2+u_2^2+u_3^2}$: módulo o magnitud.
    \item $P$: punto.
    \item $P_c$: punto de corte.
    \item $\alpha$: plano.
    \item $\vec{n}_\alpha$ vector normal de un plano. 
\end{itemize}

\section*{Operaciones con vectores}
\textbf{Producto escalar}\\
\[
    \vec{u}\cdot \vec{v}=|u|\cdot |v|\cdot \cos \theta = u_1\cdot v_1+u_2\cdot v_2+u_3\cdot v_3
\]
donde $\theta$ es el \textbf{ángulo} entre los vectores.

\section*{Planos}
\subsection*{Construcción de planos}
Para construir un plano, se deben cumplir uno de estos dos requisitos equivalentes:
\begin{itemize}
    \item 3 puntos que formen el plano
    \item 1 punto y dos vectores directores
\end{itemize}
con esto, seguiremos la siguiente fórmula para crear el plano:
\[
    \alpha \equiv \begin{vmatrix}
    u_1 & v_1 & x-P_1 \\ 
    u_2 & v_2 & y-P_2 \\ 
    u_3 & v_3 & z-P_3
    \end{vmatrix}=0
\]
\subsection*{Recta normal de un plano}
\begin{normal*}[Recta normal]
    Vector cuya dirección es perpendicular a la superficie en la que está.
\end{normal*}
Los componentes de este vector director son los coeficientes que multiplican a las
componentes $x,y$ y $z$ del plano:
\begin{align}
    \pi\equiv Ax+By+Cz+D=0\\
    \vec{n}_\pi=(A,B,C)
\end{align}
\end{document}