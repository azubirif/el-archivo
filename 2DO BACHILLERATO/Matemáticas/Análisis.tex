\documentclass{article}
\usepackage{amsmath}
\usepackage{amsthm}
\usepackage{amsfonts}
\usepackage{multicol}
\usepackage{physics}

\theoremstyle{plain}

\title{Análisis}
\date{}
\author{}

\newtheorem*{bolzano*}{Teorema de Bolzano}
\newtheorem*{rolle*}{Teorema de Rolle}

\begin{document}
\maketitle
\section*{Tabla de derivadas}

\begin{table}
    \centering
    \begin{tabular}{|c|c|}
        \hline
        y & y'\\
        \hline
        $[f]^n$  & $n\cdot f^{n-1}\cdot f'$\\
        $e^{f}$ & $e^{f}\cdot f'$ \\
        $a^f$ & $a^f\cdot \ln a \cdot f'$ \\
        $\ln f$ & $\frac{f'}{f}$ \\
        $\sin f$ & $\cos f \cdot f'$\\
        $\cos f$ & $-\sin f\cdot f'$\\
        $\tan f$ & $ \frac{f'}{1-\cos ^2 f}$\\
        $\arcsin f$ & $ \frac{f'}{\sqrt{1-f^2}}$\\
        $\arccos f$ & $ \frac{-f'}{\sqrt{1-f^2}}$\\
        $\arctan f$ & $ \frac{f'}{1+f^2}$\\    
        \hline
    \end{tabular}
\end{table}

\pagebreak
\section*{Elementos básicos}
\begin{itemize}
    \item \textbf{Dominio}: conjunto de valores de entrada que puede tomar la función y para los que está definida.
    \item \textbf{Rango}: conjunto de valores que puede dar la función.
    \item \textbf{Antiderivada}: función $F(x) \Rightarrow \dv {F(x)}{x}=f(x) \equiv \int f(x)dx = F(x)$
    \item \textbf{Definición de derivada} $f'(x)= \lim_{h \to 0} \frac{f(x+h)-f(h)}{h}$ 
\end{itemize}
\pagebreak
\section*{Continuidad}
Una función $f(x)$ es continua en un punto $x=a$ si se cumplen las \textbf{siguientes condiciones}:
\begin{enumerate}
    \item \[
        \exists f(a)
    \]
    \item \[
        \exists \lim_{x \to a} f(x) \Rightarrow \lim_{x \to a^-} f(x) = \lim_{x \to a^+}f(x)
    \]
    \item \[
        f(a)=\lim_{x \to a} f(x)
    \]
\end{enumerate}

\section*{Derivabilidad}
Una función $f(x)$ es \textbf{derivable} en un punto $x=a$ si se cumple:
\[
    \exists \lim_{x \to a} f'(x) \Rightarrow \lim_{x \to a^-} f'(x) = \lim_{x \to a^+} f'(x)
\]

\section*{Recta tangente en un punto}
La recta tangente a un punto $x=a$ se define como:
\[
    y-f(a)=f'(a)(x-a)
\]

\section*{Recta normal en un punto}
La recta normal a un punto $x=a$ se define como:
\[
    y-f(a)=- \frac{1}{f'(a)}(x-a)
\] 
\section*{Integrales}
\subsection*{Integrales por partes}
\[
    \int u\cdot dv = u\cdot v-\int v\cdot du
\]
\pagebreak

\section*{Teoremas}

\begin{bolzano*}
    Dada una función $f(x)$, continua en un intervalo $[a,b]$, si $f(a)\cdot f(b)<0$,
    podemos afirmar que $\exists c \in [a,b] / f(c)=0$.\\
    Esta fórmula se puede generalizar para afirmar que existe cualquier valor si este cumple
    que $f(a)\leq f(c)\leq f(b)$. 
\end{bolzano*}

\begin{rolle*}
    Dada una función $f(x)$, continua en un intervalo $[a,b]$, derivable en un intervalo
    $(a,b)$, si $f(a)=f(b)$, entonces podemos afirmar que $\exists c \in (a,b)/f'(c)=0$ 
\end{rolle*}

\section*{Representación gráfica}
Para representar gráficamente una función, se deben desarrollar los siguientes elementos:
\begin{itemize}
    \item Dominio de la función
    \item Puntos de corte con los ejes
    \item Asíntotas
    \item Monotonía
\end{itemize}
\pagebreak

\section*{Asíntotas}
\subsection*{Verticales}
Para indeterminaciones $ \frac{k}{0}/k \in \mathbb{R}$ 
\[
    \lim_{x \to a} f(x) \underset{\frac{k}{0}}{=} \left\{\begin{matrix}
    \lim_{x \to a^+} f(x)=\pm \infty \\ 
    \lim_{x \to a^-} f(x)=\mp \infty
    \end{matrix}\right.
\]

\subsection*{Horizontales}
Afirmamos que hay una asíntota horizontal en $\pm\infty$ si:
\[
    \lim_{x \to \pm\infty} \in \mathbb{R}
\] 

\subsection*{Oblicuas}
Para $\pm \infty$, una asíntota oblicua se expresa de la forma:
\[
    y=mx+n
\] 
siendo:

\noindent\begin{minipage}{.5\linewidth}
    \begin{equation*}
        m = \lim_{x \to \pm \infty} \frac{f(x)}{x}
    \end{equation*}
    \end{minipage}%
    \begin{minipage}{.5\linewidth}
    \begin{equation*}
        n = \lim_{x \to \pm \infty} f(x) -mx
    \end{equation*}
    \end{minipage}
\[
    m\not \in \mathbb{R} \vee n\not \in \mathbb{R} \Rightarrow \text{no hay asíntota oblicua.}
\]
\end{document}